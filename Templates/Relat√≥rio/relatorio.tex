\documentclass[twocolumn]{article}

\newcommand{\2}{\hspace{5mm}}
\newcommand{\5}{\vspace{5mm}}
\newcommand{\7}{\vspace{7mm}}
%\renewcommand{\baselinestretch}{1.4}
\textwidth=18cm
\textheight=23cm

\headheight = 12 pt
\footskip = 48 pt
\hoffset = -1.2 truecm
\voffset = -1.2 truecm

\usepackage{graphicx}

\begin{document}
\parindent 0mm
\begin{minipage}{18cm}
\begin{center}

{\Large \bf T\'{\i}tulo do relat\'orio de determinada experi\^encia} \\

\vspace{4mm}

{\large O. Respons\'avel} \\

{\it Departamento de F\'{\i}sica ?- Universidade Federal do Paran\'a 
-- Centro Polit\'ecnico -? Jd. das Am\'ericas ? 81531-990 ? Curitiba ? 
PR - Brasil \\ 
e-mail: o.responsavel@seu-provedor} \\

\vspace{1cm}

\begin{minipage}{14cm}
{\small \2 Este \'e o resumo do relat\'orio. O resumo deve ser objetivo, 
coerente e curto, com aproximadamente 100-150 palavras. Deve conter 
todas as informa\c{c}\~oes necess\'arias para que um leitor tenha uma 
id\'eia clara do que foi feito e, principalmente, quais resultados foram 
obtidos. As dicas que se seguem explicam de maneira gen\'erica como deve 
ser a formata\c{c}\~ao de um relat\'orio em estilo artigo cient\'{\i}fico, 
tanto a n\'{\i}vel de separa\c{c}\~ao l\'ogica de unidades de texto como 
em formata\c{c}\~ao de equa\c{c}\~oes, figuras e tabelas. Como pode ser 
visto, o resumo \'e formatado em um par\'agrafo simples, com tamanho de 
letra ligeiramente menor que o do texto. N\~ao se usa acr\^onimos ou 
abrevia\c{c}\~oes no resumo. \'e obrigat\'orio citar e discutir o valor 
medido das grandezas pertinentes como informa\c{c}\~ao final do resumo, 
j\'a que no caso desta disciplina as experi\^encias possuem o objetivo 
de medir alguma grandeza f\'{\i}sica bem definida.}

\end{minipage}
\end{center} 

\end{minipage}
\7

{\bf Introdu\c{c}\~ao}

\5

\2 A fun\c{c}\~ao da se\c{c}\~ao de introdu\c{c}\~ao \'e discorrer um 
pouco sobre a hist\'oria da experi\^encia em quest\~ao (quando pertinente), 
discutir os fen\^omenos e princ\'{\i}pios f\'{\i}sicos envolvidos, assim 
como realizar qualquer manipula\c{c}\~ao de conceitos e apresentar 
equa\c{c}\~oes que ser\~ao \'uteis durante a se\c{c}\~ao de an\'alise e 
discuss\~oes (ou onde forem necess\'arias). N\~ao h\'a limite para o 
tamanho da se\c{c}\~ao, isso depende da complexidade dos conceitos e dos 
c\'alculos envolvidos. 

\2 Note que a partir desta se\c{c}\~ao o artigo est\'a formatado em um 
padr\~ao de duas colunas, o que \'e seguido pela grande parte das 
revistas cient\'{\i}ficas da nossa \'area. O tamanho de letra fica o 
mesmo at\'e o final das conclus\~oes.

\2 Equa\c{c}\~oes devem ser numeradas em forma sequencial (por todo o 
relat\'orio, onde quer que apare\c{c}am) e o padr\~ao \'e geralmente o 
seguinte:

\begin{equation}
x = x_{0} + v t,
\label{1}
\end{equation}

e note que elas fazem oficialmente parte das frases, de modo que devem 
ser pontuadas adequadamente. No caso de equa\c{c}\~oes que terminem uma 
frase, a pr\'oxima senten\c{c}a \'e escrita em novo par\'agrafo, algo 
como

\begin{equation}
y = y_{0} ? v t.
\label{2}
\end{equation}

\2 Claro que aqui a prefer\^encia \'e do usu\'ario. Pode-se tentar 
escrever equa\c{c}\~oes simples como fiz acima, pode-se usar algum 
editor de equa\c{c}\~oes do Word (argh!) ou mesmo escanear e colar 
como figura (no \LaTeX tudo ? feito automaticamente). S\'o n\~ao 
esque\c{c}a a numera\c{c}\~ao seq?encial e cite adequadamente, como 
por exemplo ao comentar a eq. \ref{1} ou fazer manipula\c{c}\~oes 
como isolar a velocidade v da eq. \ref{2} e substituir em \ref{1}.

%Este ? um truque para fazer a mudan?a de p?gina e levar em conta que 
%foi definida uma minipage para o t?tulo, autor e resumo. Deve haver 
%modos mais simples hoje em dia, mas isso foi desenvolvido h? quase duas 
%d?cadas e nunca me preocupei em achar um caminho melhor porque funciona...
%Basicamente voc? compila o cara uma vez com os dois comandos abaixo em 
%modo coment?rio, v? exatamente onde o texto deve ser quebrado e insere as
%linhas de comando na posi??o adequada. Recome?a o texto sem o par?grafo (\2).
\newpage
\vspace*{8.1cm}

\2 E, finalmente, evite a todo custo fazer ?copy and paste? de sites 
de internet. Em 90\% dos casos, material provindo de internet sem uma 
cuidadosa leitura pode trazer inconsist\^encias e, freq?entemente, 
erros conceituais grosseiros. A internet \'e uma \'otima ferramenta 
mas deve ser utilizada com muito bom senso porque est\'a cheia de lixo 
(para dizer o m\'{\i}nimo...). Cuidado tamb\'em com c\'opia de 
relat\'orios de amigos, ou coisas do tipo. Aproveite a oportunidade 
de refor\c{c}ar a pr\'atica de escrever relat\'orios, vai ser uma boa 
revis\~ao e voc\^e pode aprender bastante f\'{\i}sica e hist\'oria da 
f\'{\i}sica se fizer sozinho. E \underline{jamais} d\^e um ``copy and 
paste'' na introdu\c{c}\~ao do roteiro da experi\^encia!

\2 Como sempre nos baseamos em algo para escrever a introdu\c{c}\~ao, 
as fontes devem ser corretamente referenciadas \cite{moyses}. 

\2 Este arquivo pode e deve ser usado como {\it template} para a 
confec\c{c}\~ao de seus relat\'orios, j\'a que isto pode evitar perda 
de tempo gasto com formata\c{c}\~ao. Para seu conforto e 
adequa\c{c}\~ao, h\'a tr\^es vers\~oes para diferentes editores de 
texto: \LaTeX, Open Office e, para quem ainda insiste, Word (argh!). 

\2 Pode ser que por alguma situa\c{c}\~ao inusitada um relat\'orio 
tenha que ser enviado por e-mail. Neste caso, salve o relat\'orio em 
formato .ps ou .pdf, para sua seguran\c{c}a. Por um problema de 
compatibilidade de vers\~oes ou mesmo de fontes de letra, pode ser que 
o arquivo enviado em .sxw ou .doc (argh!) perca o formato original. 
Salvando em .ps ou .pdf voc\^e garante (exceto no caso do \LaTeX, onde 
n\~ao h\'a problemas com formata\c{c}\~ao dependente de computadores) 
que o destinat\'ario receber\'a exatamente o que voc\^e enviou. Al\'em 
disso, \'e mais seguro para voc\^e, pois o destinat\'ario n\~ao teria 
como modificar o texto de maneira f\'acil.

\newpage

{\bf Procedimento experimental}

\5 

\2 Esta se\c{c}\~ao traz uma descri\c{c}\~ao do que foi realizado em 
laborat\'orio. Descreva a montagem experimental utilizada, incluindo 
detalhes que possam parecer importantes para voc\^e. Descreva a 
sistem\'atica de medidas empregada, como os dados foram obtidos. A 
id\'eia \'e que um leitor poderia reproduzir perfeitamente a 
experi\^encia que foi realizada com base nas informa\c{c}\~oes 
contidas nesta se\c{c}\~ao.

\2 Pe\c{c}o encarecidamente que voc\^e descreva o que fez, n\~ao o que 
era para ter sido feito. Saber com clareza quais foram seus passos \'e 
fundamental para compreender eventuais comportamentos estranhos de seus 
dados. Novamente, \underline{jamais} d\^e um ``copy and paste'' do roteiro!

\7

{\bf Resultados e discuss\~ao}

\5 

\2 Aqui s\~ao apresentados os dados obtidos (geralmente em tabelas), 
\underline{com suas respectivas imprecis\~oes} \underline{experimentais}. 
Assim, \'e fundamental que durante a execu\c{c}\~ao da pr\'atica voc\^e 
registre qual a precis\~ao de cada instrumento pois dever\'a levar em conta 
de maneira muito s\'eria a eventual propaga\c{c}\~ao de erros em cada um 
dos experimentos. 

\2 Dependendo do que for medido e do que for solicitado pelo roteiro, \`as 
vezes os dados s\~ao apresentados em tabelas e toma-se o valor m\'edio da 
grandeza de interesse (obrigatoriamente deve ser inclu\'{\i}do o desvio 
padr\~ao e um coment\'ario sobre a compara\c{c}\~ao do desvio padr\~ao com 
as imprecis\~oes experimentais das medidas); em outros casos a an\'alise 
\'e mais complexa e ser\'a necess\'ario fazer um ou mais gr\'aficos para 
se obter as grandezas desejadas. Novamente, \'e necess\'ario extrair dos 
gr\'aficos valores atrav\'es de ajustes lineares, exponenciais ou de outro 
tipo, e cada um dos par\^ametros do ajuste possui uma incerteza associada 
devido \`a dispers\~ao dos dados. Programas como o Origin j\'a fazem esses 
ajustes e fornecem os par\^ametros com suas respectivas incertezas de 
maneira autom\'atica. Devem ser informados e comparados com o que for 
apropriado.

\2 As legendas das tabelas s\~ao sempre colocadas \underline{em cima} das 
mesmas, e usaremos neste nosso formato uma divis\~ao livre, um exemplo da 
qual pode ser visto na tabela \ref{t1}. Por\'em \'e costume de algumas 
revistas que as divis\~oes sejam somente na horizontal, como pode ser visto 
na tabela \ref{t2}. A escolha \'e sua. O tamanho de letra das legendas 
(vale para figuras ou tabelas) deve ser um pouco menor que a letra do texto, 
para evitar confus\~ao.

\begin{table} 
\caption{\small Exemplo de divis\~ao total, com margeamento horizontal e 
vertical. Note as incertezas experimentais e a propaga\c{c}\~ao de erro, 
e o uso do n\'umero adequado de algarismos significativos.}
\begin{center}
\begin{tabular}{|c|c|c|} \hline
{\bf Posi\c{c}\~ao (m)} & {\bf Tempo (s)} & {\bf Velocidade (m/s)} \\ \hline
1.0 $\pm$ 0.1 & 11.0 $\pm$ 0.5 & 0.09 $\pm$ 0.01 \\ \hline
2.0 $\pm$ 0.1 & 20.0 $\pm$ 0.5 & 0.100 $\pm$ 0.008 \\ \hline
3.0 $\pm$ 0.1 & 29.5 $\pm$ 0.5 & 0.102 $\pm$ 0.005 \\ \hline
4.0 $\pm$ 0.1 & 40.5 $\pm$ 0.5 & 0.099 $\pm$ 0.004 \\ \hline
\end{tabular}
\end{center}
\label{t1}
\end{table}

\begin{table}
\caption{\small Exemplo de divis\~ao puramente horizontal, com linhas 
duplas para indicar in\'{\i}cio e final da tabela, nos moldes dos 
jornais da APS (Phys. Rev. e Phys. Rev. Lett.).}
\begin{center}
\begin{tabular}{ccc} \hline \hline
Posi\c{c}\~ao (m) & Tempo (s) & Velocidade (m/s) \\ \hline
1.0 $\pm$ 0.1 & 11.0 $\pm$ 0.5 & 0.09 $\pm$ 0.01 \\ 
2.0 $\pm$ 0.1 & 20.0 $\pm$ 0.5 & 0.100 $\pm$ 0.008 \\ 
3.0 $\pm$ 0.1 & 29.5 $\pm$ 0.5 & 0.102 $\pm$ 0.005 \\ 
4.0 $\pm$ 0.1 & 40.5 $\pm$ 0.5 & 0.099 $\pm$ 0.004 \\ \hline \hline
\end{tabular}
\end{center}
\label{t2}
\end{table}

\2 As legendas das figuras s\~ao sempre colocadas abaixo das mesmas, e 
com tamanho de letra menor (vide o caso das tabelas). Coloque os nomes 
e unidades corretos para as grandezas de abcissa e de ordenada. Escolha 
tamanhos de letra que sejam vis\'{\i}veis quando a figura for inserida 
na coluna, e isso vale tamb\'em para os valores das escalas. A 
numera\c{c}\~ao das figuras \'e seq?encial, como sempre (o \LaTeX ~ faz 
automaticamente). 

\begin{figure}
%\includegraphics{seno.eps}
\includegraphics[width=0.9\columnwidth,keepaspectratio]{seno.eps}
\caption{Posi\c{c}\~ao da massa M atrelada a uma mola em fun\c{c}\~ao do 
tempo.}
\end{figure}

\2 Uma vez apresentados os dados, vem a parte mais importante do 
relat\'orio: o que fazer com eles! Cada experimento pede alguma coisa 
em particular. \`as vezes um tratamento estat\'{\i}stico com m\'edias e 
desvios-padr\~ao, \`as vezes gr\'aficos dos dados experimentais com 
ajustes te\'oricos segundo alguma express\~ao conhecida (ou mesmo 
ajustes lineares simples) para extrair alguma grandeza espec\'{\i}fica. 
Geralmente os gr\'aficos facilitam ao leitor a visualiza\c{c}\~ao das 
tend\^encias das medidas, e porque foi escolhido este ou aquele m\'etodo 
de ajuste pode ser ali facilmente justificado. \'e imprescind\'{\i}vel 
que os resultados obtidos sejam discutidos \`a luz do que era esperado. 
Se h\'a determina\c{c}\~ao de alguma constante ou valor bem conhecido na 
literatura, a compara\c{c}\~ao deve ser feita. Aqui entra a import\^ancia 
das incertezas experimentais! Muitas vezes a medida pode apresentar valor 
mais alto ou mais baixo que o esperado, e a propaga\c{c}\~ao de erros vai 
dizer se isto est\'a de acordo com as limita\c{c}\~oes dos equipamentos 
ou se h\'a algum outro fator intr\'{\i}nseco que n\~ao foi levado em 
considera\c{c}\~ao e que poderia justificar a diferen\c{c}a de resultados. 
Veja que s\'o uma boa descri\c{c}\~ao do experimento pode ajudar a 
localizar prov\'aveis fontes de erros que poderiam justificar as eventuais 
diferen\c{c}as encontradas ap\'os a an\'alise de incertezas experimentais!

\7

{\bf Conclus\~oes}

\5

\2 As conclus\~oes devem abordar brevemente o experimento efetuado (n\~ao 
\'e ``copy and paste'' da introdu\c{c}\~ao ou parte dela!) e se centrarem 
nos resultados obtidos, mas de maneira resumida. A que conclus\~oes estes 
resultados levam? Como os resultados se comparam com os modelos te\'oricos 
existentes e/ou com os valores j\'a conhecidos para as grandezas 
investigadas? Houve problemas na execu\c{c}\~ao que poderiam responder por 
eventuais discrep\^ancias? Alguma sugest\~ao de melhoria?

\7

\2 Refer\^encias devem ser listadas por ordem de cita\c{c}\~ao no texto, 
no formato usual dos artigos cient\'{\i}ficos. Abaixo est\~ao exemplos 
para cita\c{c}\~oes de livros \cite{moyses} e de artigos cient\'{\i}ficos 
em revistas especializadas \cite{BCS}. Note o tamanho de letra semelhante 
aos das legendas de tabelas e figuras.

\7 

\begin{thebibliography}{99}

\bibitem{moyses} H. Moys\'es Nussenzveig, {\it Curso de F\'{\i}sica 
B\'asica}, vol. 1 ? Mec\^anica, 3a ed. (Edgard Bl?scher, S\~ao Paulo, 
1996), p. 1.

\bibitem{BCS} J. Bardeen, L. N. Cooper, and J. R. Schrieffer, {\it Phys. 
Rev.} {\bf 106}, 162 (1957); id., {\it Phys. Rev.} {\bf 108}, 1175 (1957).
 
\end{thebibliography}

\end{document}

