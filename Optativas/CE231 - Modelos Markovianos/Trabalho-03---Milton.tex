% Options for packages loaded elsewhere
\PassOptionsToPackage{unicode}{hyperref}
\PassOptionsToPackage{hyphens}{url}
%
\documentclass[
]{article}
\usepackage{lmodern}
\usepackage{amssymb,amsmath}
\usepackage{ifxetex,ifluatex}
\ifnum 0\ifxetex 1\fi\ifluatex 1\fi=0 % if pdftex
  \usepackage[T1]{fontenc}
  \usepackage[utf8]{inputenc}
  \usepackage{textcomp} % provide euro and other symbols
\else % if luatex or xetex
  \usepackage{unicode-math}
  \defaultfontfeatures{Scale=MatchLowercase}
  \defaultfontfeatures[\rmfamily]{Ligatures=TeX,Scale=1}
\fi
% Use upquote if available, for straight quotes in verbatim environments
\IfFileExists{upquote.sty}{\usepackage{upquote}}{}
\IfFileExists{microtype.sty}{% use microtype if available
  \usepackage[]{microtype}
  \UseMicrotypeSet[protrusion]{basicmath} % disable protrusion for tt fonts
}{}
\makeatletter
\@ifundefined{KOMAClassName}{% if non-KOMA class
  \IfFileExists{parskip.sty}{%
    \usepackage{parskip}
  }{% else
    \setlength{\parindent}{0pt}
    \setlength{\parskip}{6pt plus 2pt minus 1pt}}
}{% if KOMA class
  \KOMAoptions{parskip=half}}
\makeatother
\usepackage{xcolor}
\IfFileExists{xurl.sty}{\usepackage{xurl}}{} % add URL line breaks if available
\IfFileExists{bookmark.sty}{\usepackage{bookmark}}{\usepackage{hyperref}}
\hypersetup{
  pdftitle={3º CE231 - Modelos Markovianos},
  pdfauthor={XXXX},
  hidelinks,
  pdfcreator={LaTeX via pandoc}}
\urlstyle{same} % disable monospaced font for URLs
\usepackage[margin=1in]{geometry}
\usepackage{graphicx,grffile}
\makeatletter
\def\maxwidth{\ifdim\Gin@nat@width>\linewidth\linewidth\else\Gin@nat@width\fi}
\def\maxheight{\ifdim\Gin@nat@height>\textheight\textheight\else\Gin@nat@height\fi}
\makeatother
% Scale images if necessary, so that they will not overflow the page
% margins by default, and it is still possible to overwrite the defaults
% using explicit options in \includegraphics[width, height, ...]{}
\setkeys{Gin}{width=\maxwidth,height=\maxheight,keepaspectratio}
% Set default figure placement to htbp
\makeatletter
\def\fps@figure{htbp}
\makeatother
\setlength{\emergencystretch}{3em} % prevent overfull lines
\providecommand{\tightlist}{%
  \setlength{\itemsep}{0pt}\setlength{\parskip}{0pt}}
\setcounter{secnumdepth}{-\maxdimen} % remove section numbering

\title{3º CE231 - Modelos Markovianos}
\usepackage{etoolbox}
\makeatletter
\providecommand{\subtitle}[1]{% add subtitle to \maketitle
  \apptocmd{\@title}{\par {\large #1 \par}}{}{}
}
\makeatother
\subtitle{III Decomposição do espaço de estados}
\author{XXXX}
\date{17 de Agosto de 2020}

\begin{document}
\maketitle

\hypertarget{exercicio-1}{%
\subsubsection{Exercicio 1}\label{exercicio-1}}

Mostrar que se o estado \(x\) é recorrente e não se comunica com o
estado \(y_{i}\) então \(\gamma_{x,y} = 0\)

\textbf{R:} Sendo \(x\) recorrente, então \(\rho_{x,x}=1\). Dessa forma
a probabilidade de sair de \(x\) e chegar em \(y\) é 0, ou seja
\(\rho_{x,y}=0\)

\(\gamma_{x,y}^{n} = 0\) porque \(\rho_{x,y}=0\)

\emph{Outra forma de demonstrar seria:}

S(x,y)

x é um estado recorrente logo \{\(_{\rho_{x,x}=1*}^{\rho_{x,y}=0**}\)

Seja \(c_{n}\) uma cadeia de Markov com espaço de estados S

\(N(y)\) = nº vezes que a cadeia está no espaço y

\(N(y)=\sum^{\infty}_{n=1} l_{y}(C_{n})\)

também observamos que \(N(y)\geqslant 1\) é o mesmo que
\(\Gamma_{y}<\infty\) logo:

\(\rho_{x}(N(y)\geqslant 1)=\rho_{x}(\Gamma<\infty)=\rho_{x,y}\)

Sejam m e n números inteiros positivos. Sabemos que a prob com a qual a
cadeia começando em \(x\) visitar a primeira vez \(y\) no tempo m e
visitar novamente no tempo n é:

\(\rho_{x}(\Gamma_{y}=m)\rho_{y}(\Gamma_{y}=n)\)

Logo:

\(\rho(N(y)\geqslant 2) = \sum^{\infty}_{m=1}\sum^{\infty}_{n=1}\rho_{x}(\Gamma_{y}=m)\rho_{y}(\Gamma_{y}=n)=[\sum^{\infty}_{m=1}\rho_{x}(\Gamma_{y}=m)][\sum^{\infty}_{n=1}\rho_{y}(\Gamma_{y}=n)]=\)

\(\rho_{x,x}*\rho_{x,y}**=1.0=0\)

\hypertarget{exercicio-3}{%
\subsubsection{Exercicio 3}\label{exercicio-3}}

Mostre que se o estado \(x\) se comunica com \(y\) e \(y\) se comunica
com \(z\), então \(x\) se comunica com \(z\)

\textbf{R:} Pela Definição 19, temos que se \(x\) se comunica com \(y\),
então temos \(\rho_{x}(\Gamma_{y}=n)>0\) para algum \(n\) finito.
Portanto do mesmo princípio, se \(y\) se comunica com \(z\) temos
\(\rho_{y}(\Gamma_{z}=m)>0\) para algum \(m\) finito. Portanto
\(\rho_{x}(\Gamma_{z}=m+n)>0\) o que demonstra que \(x\) se comunica com
\(z\)

\hypertarget{exercicio-4}{%
\subsubsection{Exercicio 4}\label{exercicio-4}}

\[\Gamma = \begin{pmatrix}
0.0 & 0.5 & 0.0 & 0.0 & 0.5 & 0.0 & 0.0 & 0.0 & 0.0 \\
0.0 & 0.0 & 1.0 & 0.0 & 0.0 & 0.0 & 0.0 & 0.0 & 0.0 \\
0.0 & 0.0 & 0.0 & 1.0 & 0.0 & 0.0 & 0.0 & 0.0 & 0.0 \\
1.0 & 0.0 & 0.0 & 0.0 & 0.0 & 0.0 & 0.0 & 0.0 & 0.0 \\
0.0 & 0.0 & 0.0 & 0.0 & 0.0 & 1.0 & 0.0 & 0.0 & 0.0 \\
0.0 & 0.0 & 0.0 & 0.0 & 0.0 & 0.0 & 1.0 & 0.0 & 0.0 \\
0.0 & 0.0 & 0.0 & 0.0 & 0.0 & 0.0 & 0.0 & 1.0 & 0.0 \\
0.0 & 0.0 & 0.0 & 0.0 & 0.0 & 0.0 & 0.0 & 0.0 & 1.0 \\
1.0 & 0.0 & 0.0 & 0.0 & 0.0 & 0.0 & 0.0 & 0.0 & 0.0
\end{pmatrix}\]

Esta cadeia é irredutível? ou seja, prove que o conjunto de estados
irredutíveis \(F\) satisfaz \(F=S\), sendo \(S=(1,\cdots ,9)\). Prove
também que está cadeia é recorrente, ou seja, prove que cada estado em
\(S\) é recorrente

\textbf{R:}

\begin{verbatim}
Exercicio 4 
 A  9 - dimensional discrete Markov Chain defined by the following states: 
 0, 1, 2, 3, 4, 5, 6, 7, 8 
 The transition matrix  (by rows)  is defined as follows: 
  0   1 2 3   4 5 6 7 8
0 0 0.5 0 0 0.5 0 0 0 0
1 0 0.0 1 0 0.0 0 0 0 0
2 0 0.0 0 1 0.0 0 0 0 0
3 1 0.0 0 0 0.0 0 0 0 0
4 0 0.0 0 0 0.0 1 0 0 0
5 0 0.0 0 0 0.0 0 1 0 0
6 0 0.0 0 0 0.0 0 0 1 0
7 0 0.0 0 0 0.0 0 0 0 1
8 1 0.0 0 0 0.0 0 0 0 0
\end{verbatim}

\begin{verbatim}
character(0)
\end{verbatim}

\begin{verbatim}
       0   1   2   3   4   5   6   7   8
[1,] 0.2 0.1 0.1 0.1 0.1 0.1 0.1 0.1 0.1
\end{verbatim}

\begin{verbatim}
       0   1   2   3   4   5   6   7   8
[1,] 0.0 0.2 0.0 0.2 0.2 0.0 0.2 0.0 0.2
[2,] 0.4 0.0 0.2 0.0 0.0 0.2 0.0 0.2 0.0
\end{verbatim}

\begin{verbatim}
character(0)
\end{verbatim}

\begin{verbatim}
Exercicio 4  Markov chain that is composed by: 
Closed classes: 
0 1 2 3 4 5 6 7 8 
Recurrent classes: 
{0,1,2,3,4,5,6,7,8}
Transient classes: 
NONE 
The Markov chain is irreducible 
The absorbing states are: NONE
\end{verbatim}

\hypertarget{exercicio-6}{%
\subsubsection{Exercicio 6}\label{exercicio-6}}

A \textbf{Fiscalía de Mídia} identificou seis estados associados à
televisão: 0 (nunca assiste TV), 1 (assiste apenas notícias), 2 (assiste
TV com bastante frequência), 3 (viciado), 4 (em modificação de
comportamento), 5 (morte encefálica). As transições de estado para
estado podem ser modeladas como uma cadeia de Markov com a seguinte
matriz de transição:

\[\Gamma = \begin{pmatrix}
1.0 & 0.0 & 0.0 & 0.0 & 0.0 & 0.0 \\
0.5 & 0.0 & 0.5 & 0.0 & 0.0 & 0.0 \\
0.1 & 0.0 & 0.5 & 0.3 & 0.0 & 0.1 \\
0.0 & 0.0 & 0.0 & 0.7 & 0.1 & 0.2 \\
1/3 & 0.0 & 0.0 & 1/3 & 1/3 & 0.0 \\
0.0 & 0.0 & 0.0 & 0.0 & 0.0 & 1.0
\end{pmatrix}\]

\begin{quote}
\textbf{a)} Quais estados são recorrentes e quais transientes.
\end{quote}

\begin{quote}
\textbf{b)} Começando do estado 1, qual é a probabilidade de o estado 5
ser atingido antes do estado 0, ou seja, qual é a probabilidade de um
visualizador de notícias acabar com morte cerebral?
\end{quote}

\textbf{R:}

\end{document}
