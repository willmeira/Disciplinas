% Options for packages loaded elsewhere
\PassOptionsToPackage{unicode}{hyperref}
\PassOptionsToPackage{hyphens}{url}
%
\documentclass[
]{article}
\usepackage{lmodern}
\usepackage{amssymb,amsmath}
\usepackage{ifxetex,ifluatex}
\ifnum 0\ifxetex 1\fi\ifluatex 1\fi=0 % if pdftex
  \usepackage[T1]{fontenc}
  \usepackage[utf8]{inputenc}
  \usepackage{textcomp} % provide euro and other symbols
\else % if luatex or xetex
  \usepackage{unicode-math}
  \defaultfontfeatures{Scale=MatchLowercase}
  \defaultfontfeatures[\rmfamily]{Ligatures=TeX,Scale=1}
\fi
% Use upquote if available, for straight quotes in verbatim environments
\IfFileExists{upquote.sty}{\usepackage{upquote}}{}
\IfFileExists{microtype.sty}{% use microtype if available
  \usepackage[]{microtype}
  \UseMicrotypeSet[protrusion]{basicmath} % disable protrusion for tt fonts
}{}
\makeatletter
\@ifundefined{KOMAClassName}{% if non-KOMA class
  \IfFileExists{parskip.sty}{%
    \usepackage{parskip}
  }{% else
    \setlength{\parindent}{0pt}
    \setlength{\parskip}{6pt plus 2pt minus 1pt}}
}{% if KOMA class
  \KOMAoptions{parskip=half}}
\makeatother
\usepackage{xcolor}
\IfFileExists{xurl.sty}{\usepackage{xurl}}{} % add URL line breaks if available
\IfFileExists{bookmark.sty}{\usepackage{bookmark}}{\usepackage{hyperref}}
\hypersetup{
  pdftitle={3º CE231 - Modelos Markovianos},
  pdfauthor={XXXX},
  hidelinks,
  pdfcreator={LaTeX via pandoc}}
\urlstyle{same} % disable monospaced font for URLs
\usepackage[margin=1in]{geometry}
\usepackage{color}
\usepackage{fancyvrb}
\newcommand{\VerbBar}{|}
\newcommand{\VERB}{\Verb[commandchars=\\\{\}]}
\DefineVerbatimEnvironment{Highlighting}{Verbatim}{commandchars=\\\{\}}
% Add ',fontsize=\small' for more characters per line
\usepackage{framed}
\definecolor{shadecolor}{RGB}{248,248,248}
\newenvironment{Shaded}{\begin{snugshade}}{\end{snugshade}}
\newcommand{\AlertTok}[1]{\textcolor[rgb]{0.94,0.16,0.16}{#1}}
\newcommand{\AnnotationTok}[1]{\textcolor[rgb]{0.56,0.35,0.01}{\textbf{\textit{#1}}}}
\newcommand{\AttributeTok}[1]{\textcolor[rgb]{0.77,0.63,0.00}{#1}}
\newcommand{\BaseNTok}[1]{\textcolor[rgb]{0.00,0.00,0.81}{#1}}
\newcommand{\BuiltInTok}[1]{#1}
\newcommand{\CharTok}[1]{\textcolor[rgb]{0.31,0.60,0.02}{#1}}
\newcommand{\CommentTok}[1]{\textcolor[rgb]{0.56,0.35,0.01}{\textit{#1}}}
\newcommand{\CommentVarTok}[1]{\textcolor[rgb]{0.56,0.35,0.01}{\textbf{\textit{#1}}}}
\newcommand{\ConstantTok}[1]{\textcolor[rgb]{0.00,0.00,0.00}{#1}}
\newcommand{\ControlFlowTok}[1]{\textcolor[rgb]{0.13,0.29,0.53}{\textbf{#1}}}
\newcommand{\DataTypeTok}[1]{\textcolor[rgb]{0.13,0.29,0.53}{#1}}
\newcommand{\DecValTok}[1]{\textcolor[rgb]{0.00,0.00,0.81}{#1}}
\newcommand{\DocumentationTok}[1]{\textcolor[rgb]{0.56,0.35,0.01}{\textbf{\textit{#1}}}}
\newcommand{\ErrorTok}[1]{\textcolor[rgb]{0.64,0.00,0.00}{\textbf{#1}}}
\newcommand{\ExtensionTok}[1]{#1}
\newcommand{\FloatTok}[1]{\textcolor[rgb]{0.00,0.00,0.81}{#1}}
\newcommand{\FunctionTok}[1]{\textcolor[rgb]{0.00,0.00,0.00}{#1}}
\newcommand{\ImportTok}[1]{#1}
\newcommand{\InformationTok}[1]{\textcolor[rgb]{0.56,0.35,0.01}{\textbf{\textit{#1}}}}
\newcommand{\KeywordTok}[1]{\textcolor[rgb]{0.13,0.29,0.53}{\textbf{#1}}}
\newcommand{\NormalTok}[1]{#1}
\newcommand{\OperatorTok}[1]{\textcolor[rgb]{0.81,0.36,0.00}{\textbf{#1}}}
\newcommand{\OtherTok}[1]{\textcolor[rgb]{0.56,0.35,0.01}{#1}}
\newcommand{\PreprocessorTok}[1]{\textcolor[rgb]{0.56,0.35,0.01}{\textit{#1}}}
\newcommand{\RegionMarkerTok}[1]{#1}
\newcommand{\SpecialCharTok}[1]{\textcolor[rgb]{0.00,0.00,0.00}{#1}}
\newcommand{\SpecialStringTok}[1]{\textcolor[rgb]{0.31,0.60,0.02}{#1}}
\newcommand{\StringTok}[1]{\textcolor[rgb]{0.31,0.60,0.02}{#1}}
\newcommand{\VariableTok}[1]{\textcolor[rgb]{0.00,0.00,0.00}{#1}}
\newcommand{\VerbatimStringTok}[1]{\textcolor[rgb]{0.31,0.60,0.02}{#1}}
\newcommand{\WarningTok}[1]{\textcolor[rgb]{0.56,0.35,0.01}{\textbf{\textit{#1}}}}
\usepackage{graphicx,grffile}
\makeatletter
\def\maxwidth{\ifdim\Gin@nat@width>\linewidth\linewidth\else\Gin@nat@width\fi}
\def\maxheight{\ifdim\Gin@nat@height>\textheight\textheight\else\Gin@nat@height\fi}
\makeatother
% Scale images if necessary, so that they will not overflow the page
% margins by default, and it is still possible to overwrite the defaults
% using explicit options in \includegraphics[width, height, ...]{}
\setkeys{Gin}{width=\maxwidth,height=\maxheight,keepaspectratio}
% Set default figure placement to htbp
\makeatletter
\def\fps@figure{htbp}
\makeatother
\setlength{\emergencystretch}{3em} % prevent overfull lines
\providecommand{\tightlist}{%
  \setlength{\itemsep}{0pt}\setlength{\parskip}{0pt}}
\setcounter{secnumdepth}{-\maxdimen} % remove section numbering

\title{3º CE231 - Modelos Markovianos}
\usepackage{etoolbox}
\makeatletter
\providecommand{\subtitle}[1]{% add subtitle to \maketitle
  \apptocmd{\@title}{\par {\large #1 \par}}{}{}
}
\makeatother
\subtitle{III Decomposição do espaço de estados}
\author{XXXX}
\date{17 de Agosto de 2020}

\begin{document}
\maketitle

\hypertarget{exercicio-1}{%
\subsubsection{Exercicio 1}\label{exercicio-1}}

Mostrar que se o estado \(x\) é recorrente e não se comunica com o
estado \(y\), então \(γ_{x,y}=0\)

\textbf{Resolução:}

Seja \(S(x,y)\) onde \(x\) é um estado recorrente e não se comunica com
\(y\), ou seja, \(\rho_{x,x}=1 \;e \; \rho_{x,y}=0\).

Seja \(C_{n}\) uma Cadeia de Markov com espaço de estados \(S\) onde
\(N(y)\) é o número de vezes em que a cadeia permanece no espaço \(y\),
então: \[N(y)=\sum_{n=1}^{\infty}I_{y}(C_{n})\]

Também observamos que \(N(y)\geq 1\) é o mesmo que \(T_{y}<\infty\)
logo: \[P_{x}(N(y)\geq1)=P_{x}(T_{y}<\infty)=\rho_{x,y}\] Temos que a
probabilidade com a qual a cadeia começando em \(x\) visitar a primeira
vez \(y\) no tempo \(m\) e visitar novamente no tempo \(n\) (onde \(m\)
e \(n\) são inteiros positivos) é: \[P_{x}(T_{y}=m).P_{y}(T_{y}=n)\]
Logo:
\[P_{x}(N(y)\geq2)=\sum_{m=1}^{\infty}\sum_{n=1}^{\infty}P_{x}(T_{y}=m).P_{y}(T_{y}=n)\\
=\left [ \sum_{m=1}^{\infty}P_{x}(T_{y}=m) \right ].\left [ \sum_{n=1}^{\infty}P_{y}(T_{y}=m)\right ]=\rho_{x,x}\rho_{x,y}=1*0=0\]

\pagebreak

\hypertarget{exercicio-3}{%
\subsubsection{Exercicio 3}\label{exercicio-3}}

Mostre que se o estado \(x\) se comunica com \(y\) e \(y\) se comunica
com \(z\), então \(x\) se comunica com \(z\).

\textbf{Resolução:}

Se \(x\) se comunica com \(y\), então temos que \(P_{x}(T_{y}=n)>0\),
para algum \(n\) finito. Pelo mesmo princípio, se \(y\) se comunica com
\(z\), então temos que \(P_{y}(T_{z}=m)>0\), para algum \(m\) finito.

Logo \(P_{x}(T_{z}=m+n)>0\), mostrando que \(x\) se comunica com \(z\).

\pagebreak

\hypertarget{exercicio-4}{%
\subsubsection{Exercicio 4}\label{exercicio-4}}

Considere uma Cadeia de Markov com espaço de estados \{\(1,2,...,9\)\} e
matriz de probabilidades de transição

\[\Gamma = \begin{pmatrix}
0.0 & 0.5 & 0.0 & 0.0 & 0.5 & 0.0 & 0.0 & 0.0 & 0.0 \\
0.0 & 0.0 & 1.0 & 0.0 & 0.0 & 0.0 & 0.0 & 0.0 & 0.0 \\
0.0 & 0.0 & 0.0 & 1.0 & 0.0 & 0.0 & 0.0 & 0.0 & 0.0 \\
1.0 & 0.0 & 0.0 & 0.0 & 0.0 & 0.0 & 0.0 & 0.0 & 0.0 \\
0.0 & 0.0 & 0.0 & 0.0 & 0.0 & 1.0 & 0.0 & 0.0 & 0.0 \\
0.0 & 0.0 & 0.0 & 0.0 & 0.0 & 0.0 & 1.0 & 0.0 & 0.0 \\
0.0 & 0.0 & 0.0 & 0.0 & 0.0 & 0.0 & 0.0 & 1.0 & 0.0 \\
0.0 & 0.0 & 0.0 & 0.0 & 0.0 & 0.0 & 0.0 & 0.0 & 1.0 \\
1.0 & 0.0 & 0.0 & 0.0 & 0.0 & 0.0 & 0.0 & 0.0 & 0.0
\end{pmatrix}\]

Esta cadeia é irredutível? Ou seja, prove que o conjunto de estados
irredutíveis \(F\) satisfaz \(F=S\), sendo \(S=\)\{\(1,...,9\)\}. Prove
também que esta cadeia é recorrente, ou seja, prove que cada estado em
\(S\) é recorrente.

\textbf{Resolução:}

\begin{Shaded}
\begin{Highlighting}[]
\NormalTok{gamma4 <-}\StringTok{ }\KeywordTok{matrix}\NormalTok{(}\KeywordTok{c}\NormalTok{(}\DecValTok{0}\NormalTok{,}\FloatTok{0.5}\NormalTok{,}\DecValTok{0}\NormalTok{,}\DecValTok{0}\NormalTok{,}\FloatTok{0.5}\NormalTok{,}\DecValTok{0}\NormalTok{,}\DecValTok{0}\NormalTok{,}\DecValTok{0}\NormalTok{,}\DecValTok{0}\NormalTok{,}\DecValTok{0}\NormalTok{,}\DecValTok{0}\NormalTok{,}\DecValTok{1}\NormalTok{,}\DecValTok{0}\NormalTok{,}\DecValTok{0}\NormalTok{,}\DecValTok{0}\NormalTok{,}\DecValTok{0}\NormalTok{,}\DecValTok{0}\NormalTok{,}\DecValTok{0}\NormalTok{,}\DecValTok{0}\NormalTok{,}\DecValTok{0}\NormalTok{,}\DecValTok{0}\NormalTok{,}\DecValTok{1}\NormalTok{,}\DecValTok{0}\NormalTok{,}\DecValTok{0}\NormalTok{,}\DecValTok{0}\NormalTok{,}\DecValTok{0}\NormalTok{,}\DecValTok{0}\NormalTok{,}\DecValTok{1}\NormalTok{,}\KeywordTok{rep}\NormalTok{(}\DecValTok{0}\NormalTok{,}\DecValTok{8}\NormalTok{),}
                  \KeywordTok{rep}\NormalTok{(}\DecValTok{0}\NormalTok{,}\DecValTok{5}\NormalTok{),}\DecValTok{1}\NormalTok{,}\KeywordTok{rep}\NormalTok{(}\DecValTok{0}\NormalTok{,}\DecValTok{3}\NormalTok{),}\KeywordTok{rep}\NormalTok{(}\DecValTok{0}\NormalTok{,}\DecValTok{6}\NormalTok{),}\DecValTok{1}\NormalTok{,}\DecValTok{0}\NormalTok{,}\DecValTok{0}\NormalTok{,}\KeywordTok{rep}\NormalTok{(}\DecValTok{0}\NormalTok{,}\DecValTok{7}\NormalTok{),}\DecValTok{1}\NormalTok{,}\DecValTok{0}\NormalTok{,}\KeywordTok{rep}\NormalTok{(}\DecValTok{0}\NormalTok{,}\DecValTok{8}\NormalTok{),}\DecValTok{1}\NormalTok{,}\DecValTok{1}\NormalTok{,}\KeywordTok{rep}\NormalTok{(}\DecValTok{0}\NormalTok{,}\DecValTok{8}\NormalTok{)), }\DataTypeTok{byrow=}\NormalTok{T, }\DataTypeTok{ncol =} \DecValTok{9}\NormalTok{, }\DataTypeTok{dimnames=}\KeywordTok{list}\NormalTok{(}\DecValTok{1}\OperatorTok{:}\DecValTok{9}\NormalTok{))}
\NormalTok{ProbT4 =}\StringTok{ }\KeywordTok{new}\NormalTok{(}\StringTok{"markovchain"}\NormalTok{, }\DataTypeTok{states=}\KeywordTok{as.character}\NormalTok{(}\DecValTok{1}\OperatorTok{:}\DecValTok{9}\NormalTok{), }\DataTypeTok{transitionMatrix=}\NormalTok{gamma4, }\DataTypeTok{name=}\StringTok{"Gamma"}\NormalTok{)}
\end{Highlighting}
\end{Shaded}

Verificando o \emph{summary} da cadeia:

\begin{Shaded}
\begin{Highlighting}[]
\KeywordTok{summary}\NormalTok{(ProbT4)}
\end{Highlighting}
\end{Shaded}

\begin{verbatim}
Gamma  Markov chain that is composed by: 
Closed classes: 
1 2 3 4 5 6 7 8 9 
Recurrent classes: 
{1,2,3,4,5,6,7,8,9}
Transient classes: 
NONE 
The Markov chain is irreducible 
The absorbing states are: NONE
\end{verbatim}

Pela saída do R temos que todos os estados se comunicam, logo a matriz é
irredutível, com os estados sendo recorrentes, como mostra o
\textbf{Teorema 17}.

\pagebreak

\hypertarget{exercicio-6}{%
\subsubsection{Exercicio 6}\label{exercicio-6}}

\textbf{A Fiscalía de Mídia} identificou seis estados associados à
televisão: 0 (nunca assiste TV), 1 (assiste apenas notícias), 2 (assiste
TV com bastante frequência), 3 (viciado), 4 (em modificação de
comportamento), 5 (morte encefálica). As transições de estado para
estado podem ser modeladas como uma cadeia de Markov com a seguinte
matriz de transição:

\[\Gamma = \begin{pmatrix}
1.0 & 0.0 & 0.0 & 0.0 & 0.0 & 0.0 \\
0.5 & 0.0 & 0.5 & 0.0 & 0.0 & 0.0 \\
0.1 & 0.0 & 0.5 & 0.3 & 0.0 & 0.1 \\
0.0 & 0.0 & 0.0 & 0.7 & 0.1 & 0.2 \\
1/3 & 0.0 & 0.0 & 1/3 & 1/3 & 0.0 \\
0.0 & 0.0 & 0.0 & 0.0 & 0.0 & 1.0
\end{pmatrix}\]

\begin{enumerate}
\def\labelenumi{\alph{enumi})}
\tightlist
\item
  Quais estados são recorrentes e quais transientes.
\item
  Começando do estado \(1\), qual é a probabilidade de o estado \(5\)
  ser atingido antes do estado \(0\), ou seja, qual é a probabilidade de
  um visualizador de notícias acabar com morte cerebral?
\end{enumerate}

\textbf{Resolução A:} Com o auxílio do pacote \emph{markovchain}, temos
a seguinte informação:

\begin{Shaded}
\begin{Highlighting}[]
\NormalTok{estadosy <-}\StringTok{ }\KeywordTok{c}\NormalTok{(}\StringTok{"0"}\NormalTok{,}\StringTok{"1"}\NormalTok{,}\StringTok{"2"}\NormalTok{,}\StringTok{"3"}\NormalTok{,}\StringTok{"4"}\NormalTok{,}\StringTok{"5"}\NormalTok{)}
\NormalTok{y <-}\StringTok{ }\KeywordTok{matrix}\NormalTok{(}\DataTypeTok{data =} \KeywordTok{c}\NormalTok{(}\FloatTok{1.0}\NormalTok{ , }\FloatTok{0.0}\NormalTok{ , }\FloatTok{0.0}\NormalTok{ , }\FloatTok{0.0}\NormalTok{ , }\FloatTok{0.0}\NormalTok{ , }\FloatTok{0.0}\NormalTok{ ,}
\FloatTok{0.5}\NormalTok{ , }\FloatTok{0.0}\NormalTok{ , }\FloatTok{0.5}\NormalTok{ , }\FloatTok{0.0}\NormalTok{ , }\FloatTok{0.0}\NormalTok{ , }\FloatTok{0.0}\NormalTok{ ,}
\FloatTok{0.1}\NormalTok{ , }\FloatTok{0.0}\NormalTok{ , }\FloatTok{0.5}\NormalTok{ , }\FloatTok{0.3}\NormalTok{ , }\FloatTok{0.0}\NormalTok{ , }\FloatTok{0.1}\NormalTok{ ,}
\FloatTok{0.0}\NormalTok{ , }\FloatTok{0.0}\NormalTok{ , }\FloatTok{0.0}\NormalTok{ , }\FloatTok{0.7}\NormalTok{ , }\FloatTok{0.1}\NormalTok{ , }\FloatTok{0.2}\NormalTok{ ,}
\DecValTok{1}\OperatorTok{/}\DecValTok{3}\NormalTok{ , }\FloatTok{0.0}\NormalTok{ , }\FloatTok{0.0}\NormalTok{ , }\DecValTok{1}\OperatorTok{/}\DecValTok{3}\NormalTok{ , }\DecValTok{1}\OperatorTok{/}\DecValTok{3}\NormalTok{ , }\FloatTok{0.0}\NormalTok{ ,}
\FloatTok{0.0}\NormalTok{ , }\FloatTok{0.0}\NormalTok{ , }\FloatTok{0.0}\NormalTok{ , }\FloatTok{0.0}\NormalTok{ , }\FloatTok{0.0}\NormalTok{ , }\FloatTok{1.0}
\NormalTok{), }\DataTypeTok{nrow=}\DecValTok{6}\NormalTok{,}\DataTypeTok{ncol=}\DecValTok{6}\NormalTok{,}\DataTypeTok{byrow=}\NormalTok{T, }\DataTypeTok{dimnames=}\KeywordTok{list}\NormalTok{(estadosy,estadosy))}
\NormalTok{y}
\end{Highlighting}
\end{Shaded}

\begin{verbatim}
          0 1   2         3         4   5
0 1.0000000 0 0.0 0.0000000 0.0000000 0.0
1 0.5000000 0 0.5 0.0000000 0.0000000 0.0
2 0.1000000 0 0.5 0.3000000 0.0000000 0.1
3 0.0000000 0 0.0 0.7000000 0.1000000 0.2
4 0.3333333 0 0.0 0.3333333 0.3333333 0.0
5 0.0000000 0 0.0 0.0000000 0.0000000 1.0
\end{verbatim}

\begin{Shaded}
\begin{Highlighting}[]
\KeywordTok{library}\NormalTok{(markovchain)}
\NormalTok{Proby =}\StringTok{ }\KeywordTok{new}\NormalTok{(}\StringTok{"markovchain"}\NormalTok{, }\DataTypeTok{states=}\NormalTok{estadosy, }\DataTypeTok{transitionMatrix=}\NormalTok{y)}
\KeywordTok{transientStates}\NormalTok{(Proby)}
\end{Highlighting}
\end{Shaded}

\begin{verbatim}
[1] "1" "2" "3" "4"
\end{verbatim}

\begin{Shaded}
\begin{Highlighting}[]
\KeywordTok{steadyStates}\NormalTok{(Proby)}
\end{Highlighting}
\end{Shaded}

\begin{verbatim}
     0 1 2 3 4 5
[1,] 0 0 0 0 0 1
[2,] 1 0 0 0 0 0
\end{verbatim}

Logo, podemos observar que os estados 0 e 5 são recorrentes, enquanto os
estados 1, 2, 3 e 4 são transientes.

\textbf{Resolução B:}

Seja o teorema 19:

\(f(x)=\sum_{y \in F}\gamma_{x,z}+\sum_{y \in St}\gamma_{x,y}f(y),\)
\(x \in St\)

\(f(x)=\rho_{x,f},\) \(x \in St\)

Queremos a probabilidade partindo do estado ``1'' chegando ao estado
``5'' antes de chegar ao estado ``0'', então toma-se \(F\)=\{5\}.

Podemos tomar \(F\)=\{\(5,0\)\} e substrair \(\gamma_{x,0}\),
\(x \in St\)

Então:

\(f(1) = \gamma_{1,5}+\gamma_{1,1}f(1)+\gamma_{1,2}f(2)+\gamma_{1,3}f(3)+\gamma_{1,4}f(4)\)

\(f(2) = \gamma_{2,5}+\gamma_{2,1}f(1)+\gamma_{2,2}f(2)+\gamma_{2,3}f(3)+\gamma_{2,4}f(4)\)

\(f(3) = \gamma_{3,5}+\gamma_{3,1}f(1)+\gamma_{3,2}f(2)+\gamma_{3,3}f(3)+\gamma_{3,4}f(4)\)

\(f(4) = \gamma_{4,5}+\gamma_{4,1}f(1)+\gamma_{4,2}f(2)+\gamma_{4,3}f(3)+\gamma_{4,4}f(4)\)

com isso temos:

\(f(1)=0.5f(2)\)

\(f(2)=0.5f(2)+0,3f(3)+0,1\)

\(f(3)=0.7f(3)+0,1f(4)+0,2\)

\(f(4)=\frac{1}{3}f(3)+\frac{1}{3}f(4)\)

Resolvendo o sistema, temos:

\(f(1)=0,34\) \(f(2)=0,68\) \(f(3)=0,8\) \(f(4)=0,4\)

Como \(f(x)=\rho_{x,f}\),

Temos: \(f(1)=\rho_{1,5}=0,34\)

Para encontrarmos a probabilidade de que assintoticamente pelo R,
partindo de determinado estado, o visualizador atinja determinado estado
antes de entrar no estado absorvente, podemos utilizar uma potência
elevada da matriz de transição:

\begin{Shaded}
\begin{Highlighting}[]
\KeywordTok{require}\NormalTok{(expm)}
\NormalTok{y }\OperatorTok\StringTok{ }\DecValTok{10000}
\end{Highlighting}
\end{Shaded}

\begin{verbatim}
     0 1 2 3 4    5
0 1.00 0 0 0 0 0.00
1 0.66 0 0 0 0 0.34
2 0.32 0 0 0 0 0.68
3 0.20 0 0 0 0 0.80
4 0.60 0 0 0 0 0.40
5 0.00 0 0 0 0 1.00
\end{verbatim}

Portanto, podemos perceber que, partindo do estado 1, o visualizador
terá probabilidade de 0.34 de atingir o estado 5 antes que caia no
estado 0.

\end{document}
