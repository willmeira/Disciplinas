% Options for packages loaded elsewhere
\PassOptionsToPackage{unicode}{hyperref}
\PassOptionsToPackage{hyphens}{url}
%
\documentclass[
]{article}
\usepackage{lmodern}
\usepackage{amssymb,amsmath}
\usepackage{ifxetex,ifluatex}
\ifnum 0\ifxetex 1\fi\ifluatex 1\fi=0 % if pdftex
  \usepackage[T1]{fontenc}
  \usepackage[utf8]{inputenc}
  \usepackage{textcomp} % provide euro and other symbols
\else % if luatex or xetex
  \usepackage{unicode-math}
  \defaultfontfeatures{Scale=MatchLowercase}
  \defaultfontfeatures[\rmfamily]{Ligatures=TeX,Scale=1}
\fi
% Use upquote if available, for straight quotes in verbatim environments
\IfFileExists{upquote.sty}{\usepackage{upquote}}{}
\IfFileExists{microtype.sty}{% use microtype if available
  \usepackage[]{microtype}
  \UseMicrotypeSet[protrusion]{basicmath} % disable protrusion for tt fonts
}{}
\makeatletter
\@ifundefined{KOMAClassName}{% if non-KOMA class
  \IfFileExists{parskip.sty}{%
    \usepackage{parskip}
  }{% else
    \setlength{\parindent}{0pt}
    \setlength{\parskip}{6pt plus 2pt minus 1pt}}
}{% if KOMA class
  \KOMAoptions{parskip=half}}
\makeatother
\usepackage{xcolor}
\IfFileExists{xurl.sty}{\usepackage{xurl}}{} % add URL line breaks if available
\IfFileExists{bookmark.sty}{\usepackage{bookmark}}{\usepackage{hyperref}}
\hypersetup{
  pdftitle={CADEIAS DE MARKOV - TRABALHO 1},
  pdfauthor={Willian Meira Schlichta - GRR20159077},
  hidelinks,
  pdfcreator={LaTeX via pandoc}}
\urlstyle{same} % disable monospaced font for URLs
\usepackage[margin=1in]{geometry}
\usepackage{color}
\usepackage{fancyvrb}
\newcommand{\VerbBar}{|}
\newcommand{\VERB}{\Verb[commandchars=\\\{\}]}
\DefineVerbatimEnvironment{Highlighting}{Verbatim}{commandchars=\\\{\}}
% Add ',fontsize=\small' for more characters per line
\usepackage{framed}
\definecolor{shadecolor}{RGB}{248,248,248}
\newenvironment{Shaded}{\begin{snugshade}}{\end{snugshade}}
\newcommand{\AlertTok}[1]{\textcolor[rgb]{0.94,0.16,0.16}{#1}}
\newcommand{\AnnotationTok}[1]{\textcolor[rgb]{0.56,0.35,0.01}{\textbf{\textit{#1}}}}
\newcommand{\AttributeTok}[1]{\textcolor[rgb]{0.77,0.63,0.00}{#1}}
\newcommand{\BaseNTok}[1]{\textcolor[rgb]{0.00,0.00,0.81}{#1}}
\newcommand{\BuiltInTok}[1]{#1}
\newcommand{\CharTok}[1]{\textcolor[rgb]{0.31,0.60,0.02}{#1}}
\newcommand{\CommentTok}[1]{\textcolor[rgb]{0.56,0.35,0.01}{\textit{#1}}}
\newcommand{\CommentVarTok}[1]{\textcolor[rgb]{0.56,0.35,0.01}{\textbf{\textit{#1}}}}
\newcommand{\ConstantTok}[1]{\textcolor[rgb]{0.00,0.00,0.00}{#1}}
\newcommand{\ControlFlowTok}[1]{\textcolor[rgb]{0.13,0.29,0.53}{\textbf{#1}}}
\newcommand{\DataTypeTok}[1]{\textcolor[rgb]{0.13,0.29,0.53}{#1}}
\newcommand{\DecValTok}[1]{\textcolor[rgb]{0.00,0.00,0.81}{#1}}
\newcommand{\DocumentationTok}[1]{\textcolor[rgb]{0.56,0.35,0.01}{\textbf{\textit{#1}}}}
\newcommand{\ErrorTok}[1]{\textcolor[rgb]{0.64,0.00,0.00}{\textbf{#1}}}
\newcommand{\ExtensionTok}[1]{#1}
\newcommand{\FloatTok}[1]{\textcolor[rgb]{0.00,0.00,0.81}{#1}}
\newcommand{\FunctionTok}[1]{\textcolor[rgb]{0.00,0.00,0.00}{#1}}
\newcommand{\ImportTok}[1]{#1}
\newcommand{\InformationTok}[1]{\textcolor[rgb]{0.56,0.35,0.01}{\textbf{\textit{#1}}}}
\newcommand{\KeywordTok}[1]{\textcolor[rgb]{0.13,0.29,0.53}{\textbf{#1}}}
\newcommand{\NormalTok}[1]{#1}
\newcommand{\OperatorTok}[1]{\textcolor[rgb]{0.81,0.36,0.00}{\textbf{#1}}}
\newcommand{\OtherTok}[1]{\textcolor[rgb]{0.56,0.35,0.01}{#1}}
\newcommand{\PreprocessorTok}[1]{\textcolor[rgb]{0.56,0.35,0.01}{\textit{#1}}}
\newcommand{\RegionMarkerTok}[1]{#1}
\newcommand{\SpecialCharTok}[1]{\textcolor[rgb]{0.00,0.00,0.00}{#1}}
\newcommand{\SpecialStringTok}[1]{\textcolor[rgb]{0.31,0.60,0.02}{#1}}
\newcommand{\StringTok}[1]{\textcolor[rgb]{0.31,0.60,0.02}{#1}}
\newcommand{\VariableTok}[1]{\textcolor[rgb]{0.00,0.00,0.00}{#1}}
\newcommand{\VerbatimStringTok}[1]{\textcolor[rgb]{0.31,0.60,0.02}{#1}}
\newcommand{\WarningTok}[1]{\textcolor[rgb]{0.56,0.35,0.01}{\textbf{\textit{#1}}}}
\usepackage{graphicx,grffile}
\makeatletter
\def\maxwidth{\ifdim\Gin@nat@width>\linewidth\linewidth\else\Gin@nat@width\fi}
\def\maxheight{\ifdim\Gin@nat@height>\textheight\textheight\else\Gin@nat@height\fi}
\makeatother
% Scale images if necessary, so that they will not overflow the page
% margins by default, and it is still possible to overwrite the defaults
% using explicit options in \includegraphics[width, height, ...]{}
\setkeys{Gin}{width=\maxwidth,height=\maxheight,keepaspectratio}
% Set default figure placement to htbp
\makeatletter
\def\fps@figure{htbp}
\makeatother
\setlength{\emergencystretch}{3em} % prevent overfull lines
\providecommand{\tightlist}{%
  \setlength{\itemsep}{0pt}\setlength{\parskip}{0pt}}
\setcounter{secnumdepth}{-\maxdimen} % remove section numbering

\title{CADEIAS DE MARKOV - TRABALHO 1}
\author{Willian Meira Schlichta - GRR20159077}
\date{5 de agosto de 2020}

\begin{document}
\maketitle

1.Uma matriz de transição para o número de linhas telefónicas ocupadas.
Suponha que o número de linhas usadas nos tempos 1, 2, \ldots{} formem
uma Cadeia de Markov com matriz de probabilidades de transição
estacionária. Essa cadeia possui seis estados possíveis 0, 1, \ldots, 5,
onde i é o estado no qual exatamente \emph{i} linhas estão sendo usadas
em um determinado momento (i=0,1,⋯,5). Suponha que a matriz de transição
Γ seja a seguinte:

\begin{Shaded}
\begin{Highlighting}[]
\KeywordTok{library}\NormalTok{(markovchain)}
\end{Highlighting}
\end{Shaded}

\begin{verbatim}
## Package:  markovchain
## Version:  0.8.5
## Date:     2020-05-21
## BugReport: http://github.com/spedygiorgio/markovchain/issues
\end{verbatim}

\begin{Shaded}
\begin{Highlighting}[]
\NormalTok{estados =}\StringTok{ }\KeywordTok{c}\NormalTok{(}\StringTok{"Profissionais"}\NormalTok{,}\StringTok{"Qualificados"}\NormalTok{,}\StringTok{"Não Qualificados"}\NormalTok{)}
\NormalTok{Prob.T=}\KeywordTok{matrix}\NormalTok{(}\KeywordTok{c}\NormalTok{(}\FloatTok{0.8}\NormalTok{,}\FloatTok{0.1}\NormalTok{,}\FloatTok{0.1}\NormalTok{,}\FloatTok{0.6}\NormalTok{,}\FloatTok{0.2}\NormalTok{,}\FloatTok{0.2}\NormalTok{,}\FloatTok{0.5}\NormalTok{,}\FloatTok{0.25}\NormalTok{,}\FloatTok{0.25}\NormalTok{),}\DataTypeTok{nrow=}\DecValTok{3}\NormalTok{,}
              \DataTypeTok{ncol=}\DecValTok{3}\NormalTok{,}\DataTypeTok{byrow=}\NormalTok{T, }\DataTypeTok{dimnames=}\KeywordTok{list}\NormalTok{(estados,estados))}
\NormalTok{ProbT =}\StringTok{ }\KeywordTok{new}\NormalTok{(}\StringTok{"markovchain"}\NormalTok{, }\DataTypeTok{states=}\NormalTok{estados, }\DataTypeTok{transitionMatrix=}\NormalTok{Prob.T,}
            \DataTypeTok{name=}\StringTok{"Classificação de profissão de um homem"}\NormalTok{)}
\NormalTok{ProbT}
\end{Highlighting}
\end{Shaded}

\begin{verbatim}
## Classificação de profissão de um homem 
##  A  3 - dimensional discrete Markov Chain defined by the following states: 
##  Profissionais, Qualificados, Não Qualificados 
##  The transition matrix  (by rows)  is defined as follows: 
##                  Profissionais Qualificados Não Qualificados
## Profissionais              0.8         0.10             0.10
## Qualificados               0.6         0.20             0.20
## Não Qualificados           0.5         0.25             0.25
\end{verbatim}

Probabilidade dos netos a partir da matriz de transição atingirem o
estado da classificação:

\begin{Shaded}
\begin{Highlighting}[]
\NormalTok{ProbT}\OperatorTok{^}\DecValTok{2}
\end{Highlighting}
\end{Shaded}

\begin{verbatim}
## Classificação de profissão de um homem^2 
##  A  3 - dimensional discrete Markov Chain defined by the following states: 
##  Profissionais, Qualificados, Não Qualificados 
##  The transition matrix  (by rows)  is defined as follows: 
##                  Profissionais Qualificados Não Qualificados
## Profissionais            0.750       0.1250           0.1250
## Qualificados             0.700       0.1500           0.1500
## Não Qualificados         0.675       0.1625           0.1625
\end{verbatim}

Neste caso conforme solicitado no exercício temos que o neto de um neto
de um trabalhador não qualificado se tornar também um trabalhador não
qualificado e de aproximadamente 37,5\%.

2.Seja \{\(X_n\) : \(n \geq 0\)\} uma Cadeia de Markov. Mostre que

\[P(X_0=x_0|X_1=x_1,...,X_n=x_n)=P(X_0=x_0|X_1=x_1)\]

Para essa demonstração, vou utilizar outra notação, será provado que:

\[P(A_{3}|A_{1}\cap A_{2})=P(A_{3}|A_{2})\]
\[P(A_{3}|A_{1}\cap A_{2})=\frac{P(A_{1}\cap A_{2} \cap A_{3})}{P(A_{1}\cap A_{2})}\]
\[=\frac{P(A_{1}\cap A_{3}|A_{2})P(A_{1})}{P(A_{1}\cap A_{2})}\]
\[=\frac{P(A_{3}|A_{2})P(A_{1}A_{2})P(A_{2})}{P(A_{1}\cap A_{2})}\]
\[=\frac{P(A_{3}|A_{2})P(A_{1}\cap A_{2})}{P(A_{1}\cap A_{2})}\]
\[=P(A_{3}|A_{2})\] Ou seja, se dado o estado atual, os estados passados
nÃO tem influencia sobre o futuro.

\begin{enumerate}
\def\labelenumi{\arabic{enumi}.}
\setcounter{enumi}{2}
\tightlist
\item
  Uma Cadeia de Markov a três estados tem a seguinte matriz de
  probabilidades de transição:
\end{enumerate}

a)Qual é o valor aproximado de (1001,3) ? Que interpretação você dá a
esse resultado?

\begin{Shaded}
\begin{Highlighting}[]
\NormalTok{estados3 =}\StringTok{ }\KeywordTok{c}\NormalTok{(}\StringTok{"1"}\NormalTok{,}\StringTok{"2"}\NormalTok{,}\StringTok{"3"}\NormalTok{)}
\NormalTok{Prob.T3=}\KeywordTok{matrix}\NormalTok{(}\KeywordTok{c}\NormalTok{(}\FloatTok{0.4}\NormalTok{,}\FloatTok{0.5}\NormalTok{,}\FloatTok{0.1}\NormalTok{,}\FloatTok{0.4}\NormalTok{,}\FloatTok{0.5}\NormalTok{,}\FloatTok{0.1}\NormalTok{,}\FloatTok{0.4}\NormalTok{,}\FloatTok{0.5}\NormalTok{,}\FloatTok{0.1}\NormalTok{),}\DataTypeTok{nrow=}\DecValTok{3}\NormalTok{,}
              \DataTypeTok{ncol=}\DecValTok{3}\NormalTok{,}\DataTypeTok{byrow=}\NormalTok{T, }\DataTypeTok{dimnames=}\KeywordTok{list}\NormalTok{(estados3,estados3))}
\NormalTok{Prob.T3 =}\StringTok{ }\KeywordTok{new}\NormalTok{(}\StringTok{"markovchain"}\NormalTok{, }\DataTypeTok{states=}\NormalTok{estados3, }\DataTypeTok{transitionMatrix=}\NormalTok{Prob.T3,}
            \DataTypeTok{name=}\StringTok{"MATRIZ DE TRANSIÇÃO"}\NormalTok{)}
\NormalTok{aj <-}\StringTok{ }\KeywordTok{c}\NormalTok{(}\DecValTok{1}\NormalTok{,}\DecValTok{0}\NormalTok{,}\DecValTok{0}\NormalTok{)}
\NormalTok{valoraprox <-}\StringTok{ }\NormalTok{aj}\OperatorTok{*}\NormalTok{(Prob.T3}\OperatorTok{^}\DecValTok{100}\NormalTok{)}
\NormalTok{valoraprox}
\end{Highlighting}
\end{Shaded}

\begin{verbatim}
##        1   2   3
## [1,] 0.4 0.5 0.1
\end{verbatim}

Neste caso observamos que a probabilidade de estarmos no estado 3 após a
centésima interação partindo de 1 e de 10\%, definida pela matriz
estacionária. A potência da matriz de probabilidades em transição de um
estado para outro é dada pelo expoente, logo a probabilidade de ir do
estado um para o três em 100 transições é dada por essa matriz.

\begin{enumerate}
\def\labelenumi{\alph{enumi})}
\setcounter{enumi}{1}
\tightlist
\item
  Qual é a probabilidade de que após o terceiro passo a cadeia esteja no
  estado 3 se o vector de probabilidades inicial é (1/3, 1/3, 1/3)?
\end{enumerate}

\begin{Shaded}
\begin{Highlighting}[]
\NormalTok{pb <-}\StringTok{ }\KeywordTok{rep}\NormalTok{(}\DecValTok{1}\OperatorTok{/}\DecValTok{3}\NormalTok{ , }\DecValTok{3}\NormalTok{)}
\NormalTok{passo3 <-}\StringTok{ }\NormalTok{pb}\OperatorTok{*}\NormalTok{(Prob.T3}\OperatorTok{^}\DecValTok{3}\NormalTok{)}
\NormalTok{passo3}
\end{Highlighting}
\end{Shaded}

\begin{verbatim}
##        1   2   3
## [1,] 0.4 0.5 0.1
\end{verbatim}

\begin{Shaded}
\begin{Highlighting}[]
\KeywordTok{mean}\NormalTok{(passo3[,}\DecValTok{3}\NormalTok{])}
\end{Highlighting}
\end{Shaded}

\begin{verbatim}
## [1] 0.1
\end{verbatim}

Encontramos a probabilidade de aproximadamente 0,1

4.Considere como espaço de estados S = \{0, 1, \ldots{} , 6\} de uma
Cadeia de Markov com matriz de transição

\begin{Shaded}
\begin{Highlighting}[]
\NormalTok{estados4 =}\StringTok{ }\KeywordTok{c}\NormalTok{(}\StringTok{"0"}\NormalTok{, }\StringTok{"1"}\NormalTok{ , }\StringTok{"2"}\NormalTok{ , }\StringTok{"3"}\NormalTok{ , }\StringTok{"4"}\NormalTok{ , }\StringTok{"5"}\NormalTok{ , }\StringTok{"6"}\NormalTok{)}
\NormalTok{prob.T4 =}\StringTok{ }\KeywordTok{matrix}\NormalTok{ (}\KeywordTok{c}\NormalTok{(}\DecValTok{1}\OperatorTok{/}\DecValTok{2}\NormalTok{,}\DecValTok{0}\NormalTok{,}\DecValTok{1}\OperatorTok{/}\DecValTok{8}\NormalTok{,}\DecValTok{1}\OperatorTok{/}\DecValTok{4}\NormalTok{,}\DecValTok{1}\OperatorTok{/}\DecValTok{8}\NormalTok{,}\DecValTok{0}\NormalTok{,}\DecValTok{0}\NormalTok{,}\DecValTok{0}\NormalTok{,}\DecValTok{0}\NormalTok{,}\DecValTok{1}\NormalTok{,}\DecValTok{0}\NormalTok{,}\DecValTok{0}\NormalTok{,}\DecValTok{0}\NormalTok{,}\DecValTok{0}\NormalTok{,}\DecValTok{0}\NormalTok{,}\DecValTok{0}\NormalTok{,}\DecValTok{0}\NormalTok{,}\DecValTok{1}\NormalTok{,}\DecValTok{0}\NormalTok{,}\DecValTok{0}\NormalTok{,}\DecValTok{0}\NormalTok{,}\DecValTok{0}\NormalTok{,}\DecValTok{1}\NormalTok{,}\DecValTok{0}\NormalTok{,}\DecValTok{0}\NormalTok{,}\DecValTok{0}\NormalTok{,}\DecValTok{0}\NormalTok{,}\DecValTok{0}\NormalTok{,}\DecValTok{0}\NormalTok{,}\DecValTok{0}\NormalTok{,}\DecValTok{0}\NormalTok{,}\DecValTok{0}\NormalTok{,}\DecValTok{1}\OperatorTok{/}\DecValTok{2}\NormalTok{,}\DecValTok{0}\NormalTok{,}\DecValTok{1}\OperatorTok{/}\DecValTok{2}\NormalTok{,}\DecValTok{0}\NormalTok{,}\DecValTok{0}\NormalTok{,}\DecValTok{0}\NormalTok{,}\DecValTok{0}\NormalTok{,}\DecValTok{1}\OperatorTok{/}\DecValTok{2}\NormalTok{,}\DecValTok{1}\OperatorTok{/}\DecValTok{2}\NormalTok{,}\DecValTok{0}\NormalTok{,}\DecValTok{0}\NormalTok{,}\DecValTok{0}\NormalTok{,}\DecValTok{0}\NormalTok{,}\DecValTok{0}\NormalTok{,}\DecValTok{0}\NormalTok{,}\DecValTok{1}\OperatorTok{/}\DecValTok{2}\NormalTok{,}\DecValTok{1}\OperatorTok{/}\DecValTok{2}\NormalTok{),}
                \DataTypeTok{nrow=}\DecValTok{7}\NormalTok{, }\DataTypeTok{ncol=}\DecValTok{7}\NormalTok{, }\DataTypeTok{byrow=}\NormalTok{T,}
                \DataTypeTok{dimnames=}\KeywordTok{list}\NormalTok{(estados4,estados4))}
\NormalTok{prob.T4 =}\StringTok{ }\KeywordTok{new}\NormalTok{ (}\StringTok{"markovchain"}\NormalTok{, }\DataTypeTok{states=}\NormalTok{estados4, }\DataTypeTok{transitionMatrix=}\NormalTok{prob.T4, }\DataTypeTok{name=}\StringTok{"Cadeia de Markoviana"}\NormalTok{)}

\NormalTok{prob.T4}
\end{Highlighting}
\end{Shaded}

\begin{verbatim}
## Cadeia de Markoviana 
##  A  7 - dimensional discrete Markov Chain defined by the following states: 
##  0, 1, 2, 3, 4, 5, 6 
##  The transition matrix  (by rows)  is defined as follows: 
##     0 1     2    3     4   5   6
## 0 0.5 0 0.125 0.25 0.125 0.0 0.0
## 1 0.0 0 1.000 0.00 0.000 0.0 0.0
## 2 0.0 0 0.000 1.00 0.000 0.0 0.0
## 3 0.0 1 0.000 0.00 0.000 0.0 0.0
## 4 0.0 0 0.000 0.00 0.500 0.0 0.5
## 5 0.0 0 0.000 0.00 0.500 0.5 0.0
## 6 0.0 0 0.000 0.00 0.000 0.5 0.5
\end{verbatim}

a)Determine quais estados são transientes e quais recorrentes

\begin{Shaded}
\begin{Highlighting}[]
\KeywordTok{transientStates}\NormalTok{(prob.T4)}
\end{Highlighting}
\end{Shaded}

\begin{verbatim}
## [1] "0"
\end{verbatim}

\begin{Shaded}
\begin{Highlighting}[]
\KeywordTok{steadyStates}\NormalTok{(prob.T4)}
\end{Highlighting}
\end{Shaded}

\begin{verbatim}
##      0         1         2         3         4         5         6
## [1,] 0 0.0000000 0.0000000 0.0000000 0.3333333 0.3333333 0.3333333
## [2,] 0 0.3333333 0.3333333 0.3333333 0.0000000 0.0000000 0.0000000
\end{verbatim}

Observamos que os estados de 1 a 6 são recorrentes, restando apenas o
estado 0 como trasiente

b)Encontre \(\rho_{0,y}\) para y=0,\ldots,6.

\begin{Shaded}
\begin{Highlighting}[]
\KeywordTok{is.accessible}\NormalTok{(}\DataTypeTok{object =}\NormalTok{ prob.T4, }\DataTypeTok{from =} \StringTok{"0"}\NormalTok{, }\DataTypeTok{to =} \StringTok{"0"}\NormalTok{)}
\end{Highlighting}
\end{Shaded}

\begin{verbatim}
## [1] TRUE
\end{verbatim}

\begin{Shaded}
\begin{Highlighting}[]
\KeywordTok{is.accessible}\NormalTok{(}\DataTypeTok{object =}\NormalTok{ prob.T4, }\DataTypeTok{from =} \StringTok{"0"}\NormalTok{, }\DataTypeTok{to =} \StringTok{"1"}\NormalTok{)}
\end{Highlighting}
\end{Shaded}

\begin{verbatim}
## [1] TRUE
\end{verbatim}

\begin{Shaded}
\begin{Highlighting}[]
\KeywordTok{is.accessible}\NormalTok{(}\DataTypeTok{object =}\NormalTok{ prob.T4, }\DataTypeTok{from =} \StringTok{"0"}\NormalTok{, }\DataTypeTok{to =} \StringTok{"2"}\NormalTok{)}
\end{Highlighting}
\end{Shaded}

\begin{verbatim}
## [1] TRUE
\end{verbatim}

\begin{Shaded}
\begin{Highlighting}[]
\KeywordTok{is.accessible}\NormalTok{(}\DataTypeTok{object =}\NormalTok{ prob.T4, }\DataTypeTok{from =} \StringTok{"0"}\NormalTok{, }\DataTypeTok{to =} \StringTok{"3"}\NormalTok{)}
\end{Highlighting}
\end{Shaded}

\begin{verbatim}
## [1] TRUE
\end{verbatim}

\begin{Shaded}
\begin{Highlighting}[]
\KeywordTok{is.accessible}\NormalTok{(}\DataTypeTok{object =}\NormalTok{ prob.T4, }\DataTypeTok{from =} \StringTok{"0"}\NormalTok{, }\DataTypeTok{to =} \StringTok{"4"}\NormalTok{)}
\end{Highlighting}
\end{Shaded}

\begin{verbatim}
## [1] TRUE
\end{verbatim}

\begin{Shaded}
\begin{Highlighting}[]
\KeywordTok{is.accessible}\NormalTok{(}\DataTypeTok{object =}\NormalTok{ prob.T4, }\DataTypeTok{from =} \StringTok{"0"}\NormalTok{, }\DataTypeTok{to =} \StringTok{"5"}\NormalTok{)}
\end{Highlighting}
\end{Shaded}

\begin{verbatim}
## [1] TRUE
\end{verbatim}

\begin{Shaded}
\begin{Highlighting}[]
\KeywordTok{is.accessible}\NormalTok{(}\DataTypeTok{object =}\NormalTok{ prob.T4, }\DataTypeTok{from =} \StringTok{"0"}\NormalTok{, }\DataTypeTok{to =} \StringTok{"6"}\NormalTok{)}
\end{Highlighting}
\end{Shaded}

\begin{verbatim}
## [1] TRUE
\end{verbatim}

\begin{Shaded}
\begin{Highlighting}[]
\KeywordTok{firstPassage}\NormalTok{(prob.T4, }\DecValTok{0}\NormalTok{, }\DecValTok{6}\NormalTok{)}
\end{Highlighting}
\end{Shaded}

\begin{verbatim}
##     0       1          2        3          4         5          6
## 1 0.5 0.00000 0.12500000 0.250000 0.12500000 0.0000000 0.00000000
## 2 0.0 0.25000 0.06250000 0.250000 0.06250000 0.0000000 0.06250000
## 3 0.0 0.25000 0.28125000 0.125000 0.03125000 0.0312500 0.06250000
## 4 0.0 0.12500 0.14062500 0.062500 0.01562500 0.0468750 0.04687500
## 5 0.0 0.06250 0.07031250 0.031250 0.00781250 0.0468750 0.03125000
## 6 0.0 0.03125 0.03515625 0.015625 0.00390625 0.0390625 0.01953125
\end{verbatim}

5.Num estudo com homens criminosos em Filadélfia descobriram que a
probabilidade de que um tipo de ataque seja seguido por um outro tipo
pode ser descrito pela seguinte matriz de transição.

\begin{Shaded}
\begin{Highlighting}[]
\NormalTok{estados5 =}\StringTok{ }\KeywordTok{c}\NormalTok{(}\StringTok{"Outro"}\NormalTok{, }\StringTok{"Injúria" , "}\NormalTok{Roubo}\StringTok{" , "}\NormalTok{Dano}\StringTok{" , "}\NormalTok{Misto}\StringTok{")}
\StringTok{prob.T5 = matrix (c(0.645,0.099,0.152,0.033,0.071,0.611,0.138,0.128,0.033,0.090,0.514,0.067,0.271,0.030,0.118,0.609,0.107,0.178,0.064,0.042,0.523,0.093,0.183,0.022,0.179),}
\StringTok{                nrow=5, ncol=5, byrow=T,}
\StringTok{                dimnames=list(estados5,estados5))}
\StringTok{prob.T5 = new ("}\NormalTok{markovchain}\StringTok{", states=estados5, transitionMatrix=prob.T5, name="}\NormalTok{Homens Criminosos}\StringTok{")}

\StringTok{prob.T5}
\end{Highlighting}
\end{Shaded}

\begin{verbatim}
## Homens Criminosos 
##  A  5 - dimensional discrete Markov Chain defined by the following states: 
##  Outro, Injúria, Roubo, Dano, Misto 
##  The transition matrix  (by rows)  is defined as follows: 
##         Outro Injúria Roubo  Dano Misto
## Outro   0.645   0.099 0.152 0.033 0.071
## Injúria 0.611   0.138 0.128 0.033 0.090
## Roubo   0.514   0.067 0.271 0.030 0.118
## Dano    0.609   0.107 0.178 0.064 0.042
## Misto   0.523   0.093 0.183 0.022 0.179
\end{verbatim}

a)Para um criminoso que comete roubo, qual é a probabilidade que o seu
próximo crime também seja um roubo?

\begin{Shaded}
\begin{Highlighting}[]
\NormalTok{roubou1 <-}\StringTok{ }\NormalTok{prob.T5}\OperatorTok{^}\DecValTok{1}
\NormalTok{roubo <-}\StringTok{ }\FloatTok{0.271}
\NormalTok{roubo}
\end{Highlighting}
\end{Shaded}

\begin{verbatim}
## [1] 0.271
\end{verbatim}

Conforme visto na tabela acima encontramos a probabilidade de
aproximadamente 27\%

b)Para um criminoso que comete roubo, qual é a probabilidade de que seu
segundo crime depois do atual também seja um roubo?

\begin{Shaded}
\begin{Highlighting}[]
\NormalTok{roubos2<-}\StringTok{ }\NormalTok{prob.T5}\OperatorTok{^}\DecValTok{2}
\NormalTok{roubos2}
\end{Highlighting}
\end{Shaded}

\begin{verbatim}
## Homens Criminosos^2 
##  A  5 - dimensional discrete Markov Chain defined by the following states: 
##  Outro, Injúria, Roubo, Dano, Misto 
##  The transition matrix  (by rows)  is defined as follows: 
##            Outro  Injúria    Roubo     Dano    Misto
## Outro   0.611872 0.097835 0.170771 0.032786 0.086736
## Injúria 0.611372 0.100010 0.167568 0.032649 0.088401
## Roubo   0.591745 0.092473 0.187079 0.031819 0.096884
## Dano    0.610616 0.097737 0.173580 0.033988 0.084079
## Misto   0.595235 0.095873 0.177666 0.031164 0.100062
\end{verbatim}

Encontramos a probabilidade de aproximadamente 19\%

c)Se essas tendências continuarem, quais são as probabilidades de longo
prazo para cada tipo de crime?

\begin{Shaded}
\begin{Highlighting}[]
\NormalTok{longoprazo <-}\StringTok{ }\KeywordTok{steadyStates}\NormalTok{(prob.T5)}
\NormalTok{longoprazo}
\end{Highlighting}
\end{Shaded}

\begin{verbatim}
##          Outro    Injúria     Roubo      Dano     Misto
## [1,] 0.6067869 0.09693348 0.1740085 0.0324979 0.0897732
\end{verbatim}

6.Considere uma Cadeia de Markov com espaço de estados S=\{0,1,2\} e
matriz de probabilidades de transição a)Mostre que esta cadeia tem uma
única distribuição estacionária \(\pi\) e encontre-a.

\begin{Shaded}
\begin{Highlighting}[]
\NormalTok{estados6 =}\StringTok{ }\KeywordTok{c}\NormalTok{(}\StringTok{"0"}\NormalTok{, }\StringTok{"1"}\NormalTok{ , }\StringTok{"2"}\NormalTok{)}
\NormalTok{prob6 =}\StringTok{ }\KeywordTok{matrix}\NormalTok{ (}\KeywordTok{c}\NormalTok{(}\FloatTok{0.4}\NormalTok{,}\FloatTok{0.4}\NormalTok{,}\FloatTok{0.2}\NormalTok{,}\FloatTok{0.3}\NormalTok{,}\FloatTok{0.4}\NormalTok{,}\FloatTok{0.3}\NormalTok{,}\FloatTok{0.2}\NormalTok{,}\FloatTok{0.4}\NormalTok{,}\FloatTok{0.4}\NormalTok{),}
                \DataTypeTok{nrow=}\DecValTok{3}\NormalTok{, }\DataTypeTok{ncol=}\DecValTok{3}\NormalTok{, }\DataTypeTok{byrow=}\NormalTok{T,}
                \DataTypeTok{dimnames=}\KeywordTok{list}\NormalTok{(estados6,estados6))}
\NormalTok{prob6 =}\StringTok{ }\KeywordTok{new}\NormalTok{ (}\StringTok{"markovchain"}\NormalTok{, }\DataTypeTok{states=}\NormalTok{estados6, }\DataTypeTok{transitionMatrix=}\NormalTok{prob6, }\DataTypeTok{name=}\StringTok{"Cadeia de Markov 3"}\NormalTok{)}

\NormalTok{prob6}
\end{Highlighting}
\end{Shaded}

\begin{verbatim}
## Cadeia de Markov 3 
##  A  3 - dimensional discrete Markov Chain defined by the following states: 
##  0, 1, 2 
##  The transition matrix  (by rows)  is defined as follows: 
##     0   1   2
## 0 0.4 0.4 0.2
## 1 0.3 0.4 0.3
## 2 0.2 0.4 0.4
\end{verbatim}

\begin{Shaded}
\begin{Highlighting}[]
\KeywordTok{steadyStates}\NormalTok{(prob6)}
\end{Highlighting}
\end{Shaded}

\begin{verbatim}
##        0   1   2
## [1,] 0.3 0.4 0.3
\end{verbatim}

Podemos observar a distribuição estacionária sendo 0,3 / 0,4 / 0,3 para
os estados 1 / 2 / 3 respectivamente.

7.Considere uma Cadeia de Markov sendo S=\{0,1,2,3,4\} o espaço de
estados e com matriz de probabilidades de transição

\begin{Shaded}
\begin{Highlighting}[]
\NormalTok{estados7 =}\StringTok{ }\KeywordTok{c}\NormalTok{(}\StringTok{"0"}\NormalTok{, }\StringTok{"1"}\NormalTok{ , }\StringTok{"2"}\NormalTok{ , }\StringTok{"3"}\NormalTok{ , }\StringTok{"4"}\NormalTok{)}
\NormalTok{prob7 =}\StringTok{ }\KeywordTok{matrix}\NormalTok{ (}\KeywordTok{c}\NormalTok{(}\DecValTok{0}\NormalTok{,}\DecValTok{1}\OperatorTok{/}\DecValTok{3}\NormalTok{,}\DecValTok{2}\OperatorTok{/}\DecValTok{3}\NormalTok{,}\DecValTok{0}\NormalTok{,}\DecValTok{0}\NormalTok{,}\DecValTok{0}\NormalTok{,}\DecValTok{0}\NormalTok{,}\DecValTok{0}\NormalTok{,}\DecValTok{1}\OperatorTok{/}\DecValTok{4}\NormalTok{,}\DecValTok{3}\OperatorTok{/}\DecValTok{4}\NormalTok{,}\DecValTok{0}\NormalTok{,}\DecValTok{0}\NormalTok{,}\DecValTok{0}\NormalTok{,}\DecValTok{1}\OperatorTok{/}\DecValTok{4}\NormalTok{,}\DecValTok{3}\OperatorTok{/}\DecValTok{4}\NormalTok{,}\DecValTok{1}\NormalTok{,}\DecValTok{0}\NormalTok{,}\DecValTok{0}\NormalTok{,}\DecValTok{0}\NormalTok{,}\DecValTok{0}\NormalTok{,}\DecValTok{1}\NormalTok{,}\DecValTok{0}\NormalTok{,}\DecValTok{0}\NormalTok{,}\DecValTok{0}\NormalTok{,}\DecValTok{0}\NormalTok{),}
                \DataTypeTok{nrow=}\DecValTok{5}\NormalTok{, }\DataTypeTok{ncol=}\DecValTok{5}\NormalTok{, }\DataTypeTok{byrow=}\NormalTok{T,}
                \DataTypeTok{dimnames=}\KeywordTok{list}\NormalTok{(estados7,estados7))}
\NormalTok{prob7 =}\StringTok{ }\KeywordTok{new}\NormalTok{ (}\StringTok{"markovchain"}\NormalTok{, }\DataTypeTok{states=}\NormalTok{estados7, }\DataTypeTok{transitionMatrix=}\NormalTok{prob7, }\DataTypeTok{name=}\StringTok{"Matriz de Transição VII"}\NormalTok{)}

\NormalTok{prob7}
\end{Highlighting}
\end{Shaded}

\begin{verbatim}
## Matriz de Transição VII 
##  A  5 - dimensional discrete Markov Chain defined by the following states: 
##  0, 1, 2, 3, 4 
##  The transition matrix  (by rows)  is defined as follows: 
##   0         1         2    3    4
## 0 0 0.3333333 0.6666667 0.00 0.00
## 1 0 0.0000000 0.0000000 0.25 0.75
## 2 0 0.0000000 0.0000000 0.25 0.75
## 3 1 0.0000000 0.0000000 0.00 0.00
## 4 1 0.0000000 0.0000000 0.00 0.00
\end{verbatim}

a)Mostre que esta é uma cadeia irredutível

\begin{verbatim}
## [1] TRUE
\end{verbatim}

\begin{verbatim}
## Matriz de Transição VII 
##  A  5 - dimensional discrete Markov Chain defined by the following states: 
##  0, 1, 2, 3, 4 
##  The transition matrix  (by rows)  is defined as follows: 
##   0         1         2    3    4
## 0 0 0.3333333 0.6666667 0.00 0.00
## 1 0 0.0000000 0.0000000 0.25 0.75
## 2 0 0.0000000 0.0000000 0.25 0.75
## 3 1 0.0000000 0.0000000 0.00 0.00
## 4 1 0.0000000 0.0000000 0.00 0.00
\end{verbatim}

b)Encontre o período

\begin{verbatim}
## [1] 3
\end{verbatim}

c)Encontre a distribuição estacionária

\begin{verbatim}
##              0         1         2          3    4
## [1,] 0.3333333 0.1111111 0.2222222 0.08333333 0.25
\end{verbatim}

\end{document}
