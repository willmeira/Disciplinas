% Options for packages loaded elsewhere
\PassOptionsToPackage{unicode}{hyperref}
\PassOptionsToPackage{hyphens}{url}
%
\documentclass[
]{article}
\usepackage{lmodern}
\usepackage{amssymb,amsmath}
\usepackage{ifxetex,ifluatex}
\ifnum 0\ifxetex 1\fi\ifluatex 1\fi=0 % if pdftex
  \usepackage[T1]{fontenc}
  \usepackage[utf8]{inputenc}
  \usepackage{textcomp} % provide euro and other symbols
\else % if luatex or xetex
  \usepackage{unicode-math}
  \defaultfontfeatures{Scale=MatchLowercase}
  \defaultfontfeatures[\rmfamily]{Ligatures=TeX,Scale=1}
\fi
% Use upquote if available, for straight quotes in verbatim environments
\IfFileExists{upquote.sty}{\usepackage{upquote}}{}
\IfFileExists{microtype.sty}{% use microtype if available
  \usepackage[]{microtype}
  \UseMicrotypeSet[protrusion]{basicmath} % disable protrusion for tt fonts
}{}
\makeatletter
\@ifundefined{KOMAClassName}{% if non-KOMA class
  \IfFileExists{parskip.sty}{%
    \usepackage{parskip}
  }{% else
    \setlength{\parindent}{0pt}
    \setlength{\parskip}{6pt plus 2pt minus 1pt}}
}{% if KOMA class
  \KOMAoptions{parskip=half}}
\makeatother
\usepackage{xcolor}
\IfFileExists{xurl.sty}{\usepackage{xurl}}{} % add URL line breaks if available
\IfFileExists{bookmark.sty}{\usepackage{bookmark}}{\usepackage{hyperref}}
\hypersetup{
  pdftitle={4º CE231 - Modelos Markovianos},
  pdfauthor={XXXX},
  hidelinks,
  pdfcreator={LaTeX via pandoc}}
\urlstyle{same} % disable monospaced font for URLs
\usepackage[margin=1in]{geometry}
\usepackage{color}
\usepackage{fancyvrb}
\newcommand{\VerbBar}{|}
\newcommand{\VERB}{\Verb[commandchars=\\\{\}]}
\DefineVerbatimEnvironment{Highlighting}{Verbatim}{commandchars=\\\{\}}
% Add ',fontsize=\small' for more characters per line
\usepackage{framed}
\definecolor{shadecolor}{RGB}{248,248,248}
\newenvironment{Shaded}{\begin{snugshade}}{\end{snugshade}}
\newcommand{\AlertTok}[1]{\textcolor[rgb]{0.94,0.16,0.16}{#1}}
\newcommand{\AnnotationTok}[1]{\textcolor[rgb]{0.56,0.35,0.01}{\textbf{\textit{#1}}}}
\newcommand{\AttributeTok}[1]{\textcolor[rgb]{0.77,0.63,0.00}{#1}}
\newcommand{\BaseNTok}[1]{\textcolor[rgb]{0.00,0.00,0.81}{#1}}
\newcommand{\BuiltInTok}[1]{#1}
\newcommand{\CharTok}[1]{\textcolor[rgb]{0.31,0.60,0.02}{#1}}
\newcommand{\CommentTok}[1]{\textcolor[rgb]{0.56,0.35,0.01}{\textit{#1}}}
\newcommand{\CommentVarTok}[1]{\textcolor[rgb]{0.56,0.35,0.01}{\textbf{\textit{#1}}}}
\newcommand{\ConstantTok}[1]{\textcolor[rgb]{0.00,0.00,0.00}{#1}}
\newcommand{\ControlFlowTok}[1]{\textcolor[rgb]{0.13,0.29,0.53}{\textbf{#1}}}
\newcommand{\DataTypeTok}[1]{\textcolor[rgb]{0.13,0.29,0.53}{#1}}
\newcommand{\DecValTok}[1]{\textcolor[rgb]{0.00,0.00,0.81}{#1}}
\newcommand{\DocumentationTok}[1]{\textcolor[rgb]{0.56,0.35,0.01}{\textbf{\textit{#1}}}}
\newcommand{\ErrorTok}[1]{\textcolor[rgb]{0.64,0.00,0.00}{\textbf{#1}}}
\newcommand{\ExtensionTok}[1]{#1}
\newcommand{\FloatTok}[1]{\textcolor[rgb]{0.00,0.00,0.81}{#1}}
\newcommand{\FunctionTok}[1]{\textcolor[rgb]{0.00,0.00,0.00}{#1}}
\newcommand{\ImportTok}[1]{#1}
\newcommand{\InformationTok}[1]{\textcolor[rgb]{0.56,0.35,0.01}{\textbf{\textit{#1}}}}
\newcommand{\KeywordTok}[1]{\textcolor[rgb]{0.13,0.29,0.53}{\textbf{#1}}}
\newcommand{\NormalTok}[1]{#1}
\newcommand{\OperatorTok}[1]{\textcolor[rgb]{0.81,0.36,0.00}{\textbf{#1}}}
\newcommand{\OtherTok}[1]{\textcolor[rgb]{0.56,0.35,0.01}{#1}}
\newcommand{\PreprocessorTok}[1]{\textcolor[rgb]{0.56,0.35,0.01}{\textit{#1}}}
\newcommand{\RegionMarkerTok}[1]{#1}
\newcommand{\SpecialCharTok}[1]{\textcolor[rgb]{0.00,0.00,0.00}{#1}}
\newcommand{\SpecialStringTok}[1]{\textcolor[rgb]{0.31,0.60,0.02}{#1}}
\newcommand{\StringTok}[1]{\textcolor[rgb]{0.31,0.60,0.02}{#1}}
\newcommand{\VariableTok}[1]{\textcolor[rgb]{0.00,0.00,0.00}{#1}}
\newcommand{\VerbatimStringTok}[1]{\textcolor[rgb]{0.31,0.60,0.02}{#1}}
\newcommand{\WarningTok}[1]{\textcolor[rgb]{0.56,0.35,0.01}{\textbf{\textit{#1}}}}
\usepackage{graphicx,grffile}
\makeatletter
\def\maxwidth{\ifdim\Gin@nat@width>\linewidth\linewidth\else\Gin@nat@width\fi}
\def\maxheight{\ifdim\Gin@nat@height>\textheight\textheight\else\Gin@nat@height\fi}
\makeatother
% Scale images if necessary, so that they will not overflow the page
% margins by default, and it is still possible to overwrite the defaults
% using explicit options in \includegraphics[width, height, ...]{}
\setkeys{Gin}{width=\maxwidth,height=\maxheight,keepaspectratio}
% Set default figure placement to htbp
\makeatletter
\def\fps@figure{htbp}
\makeatother
\setlength{\emergencystretch}{3em} % prevent overfull lines
\providecommand{\tightlist}{%
  \setlength{\itemsep}{0pt}\setlength{\parskip}{0pt}}
\setcounter{secnumdepth}{-\maxdimen} % remove section numbering

\title{4º CE231 - Modelos Markovianos}
\usepackage{etoolbox}
\makeatletter
\providecommand{\subtitle}[1]{% add subtitle to \maketitle
  \apptocmd{\@title}{\par {\large #1 \par}}{}{}
}
\makeatother
\subtitle{IV Distribuição estacionária}
\author{XXXX}
\date{17 de Agosto de 2020}

\begin{document}
\maketitle

\begin{Shaded}
\begin{Highlighting}[]
\CommentTok{##LIMPESA DO HISTORICO DO R}
\KeywordTok{rm}\NormalTok{(}\DataTypeTok{list =} \KeywordTok{ls}\NormalTok{())}

\CommentTok{##PACOTES USADOS}
\KeywordTok{library}\NormalTok{(dplyr)}
\KeywordTok{library}\NormalTok{(markovchain)}
\end{Highlighting}
\end{Shaded}

\hypertarget{exercicio-3}{%
\subsubsection{Exercicio 3}\label{exercicio-3}}

Uma Matriz de probabilidade de transição \(\Gamma\) é dita ser
duplamente estocástica e a soma por colunas é também 1, isto é, se

\[\sum_{x\epsilon S} \gamma_{x,y}=1,\forall\epsilon S\] Considere uma
cadeia com matriz de transição duplamente estocástica irredutível,
aperiódica e consistindo de \(M +1\) estados, do qual
\(S=\)\{0,1,2,\ldots,\(M\)\}. Prove que a distribuição estacionária é
dada por

\[\pi(y)=\frac{1}{M+1},y\epsilon S\]

\textbf{R:}

O enunciado mostra que a cadeia em análise é irredutível (ou seja, todos
os estados se comunicam entre si) e aperiódica, ou seja, \(d=1\)
(Definição 34). Em conjunto, essas duas afirmações evidenciam que nenhum
dos elementos da matriz de transição é nulo. Em seguida, temos a
informação de que a matriz de transição é duplamente estocástica, ou
seja, todas as colunas e linhas somam 1. Tomemos o exemplo mais simples,
para \(M = 1\), implicando uma cadeia de Markov com dois estados:

\[\Gamma = 
\begin{pmatrix}
a & b \\
c & d    
\end{pmatrix}\] Logo, temos um sistema em que \(a + b = 1\),
\(a + c = 1\), \(b + d = 1\) e \(c + d = 1\). Não é difícil perceber que
a = b = c = d, de forma que para que as condições acima mencionadas
procedam, a = b = c = d = 0.5. Para o caso de \(M = 2\), temos a
seguinte matriz de transição:

\[\Gamma = 
\begin{pmatrix}
a & b & c  \\
d & e & f  \\
g & h & i  
\end{pmatrix}\]

Da mesma forma, temos que \(a + b + c = 1\), \(d + e + f = 1\),
\(g + h + i = f\), \(a + d + g = 1\), \(b + e + h = 1\) e
\(c + f + i = 1\). A solução dessas equações implica que
\(a = b = c = d = e = f = g = h = i\), e com as condições mencionadas no
enunciado, todos os valores são iguais. Portanto,
\(a = b = c = d = e = f = g = h = i = 1/3\). Calculando a distribuição
estacionária para essas duas matrizes, temos:

\begin{Shaded}
\begin{Highlighting}[]
\NormalTok{estados3a <-}\StringTok{ }\KeywordTok{c}\NormalTok{(}\StringTok{"0"}\NormalTok{, }\StringTok{"1"}\NormalTok{)}
\NormalTok{M <-}\StringTok{ }\KeywordTok{matrix}\NormalTok{(}\DataTypeTok{data =} \KeywordTok{c}\NormalTok{(}\FloatTok{0.5}\NormalTok{, }\FloatTok{0.5}\NormalTok{, }\FloatTok{0.5}\NormalTok{, }\FloatTok{0.5}\NormalTok{), }\DataTypeTok{nrow =} \DecValTok{2}\NormalTok{, }\DataTypeTok{ncol =} \DecValTok{2}\NormalTok{,}
\DataTypeTok{byrow =} \OtherTok{TRUE}\NormalTok{, }\DataTypeTok{dimnames =} \KeywordTok{list}\NormalTok{(estados3a, estados3a));M}
\end{Highlighting}
\end{Shaded}

\begin{verbatim}
    0   1
0 0.5 0.5
1 0.5 0.5
\end{verbatim}

\begin{Shaded}
\begin{Highlighting}[]
\NormalTok{estados3b <-}\StringTok{ }\KeywordTok{c}\NormalTok{(}\StringTok{"0"}\NormalTok{, }\StringTok{"1"}\NormalTok{, }\StringTok{"2"}\NormalTok{)}
\NormalTok{Mb <-}\StringTok{ }\KeywordTok{matrix}\NormalTok{(}\DataTypeTok{data =} \KeywordTok{c}\NormalTok{(}\DecValTok{1}\OperatorTok{/}\DecValTok{3}\NormalTok{, }\DecValTok{1}\OperatorTok{/}\DecValTok{3}\NormalTok{, }\DecValTok{1}\OperatorTok{/}\DecValTok{3}\NormalTok{, }\DecValTok{1}\OperatorTok{/}\DecValTok{3}\NormalTok{, }\DecValTok{1}\OperatorTok{/}\DecValTok{3}\NormalTok{, }\DecValTok{1}\OperatorTok{/}\DecValTok{3}\NormalTok{, }\DecValTok{1}\OperatorTok{/}\DecValTok{3}\NormalTok{, }\DecValTok{1}\OperatorTok{/}\DecValTok{3}\NormalTok{, }\DecValTok{1}\OperatorTok{/}\DecValTok{3}\NormalTok{), }\DataTypeTok{nrow =} \DecValTok{3}\NormalTok{, }\DataTypeTok{ncol =} \DecValTok{3}\NormalTok{,}
\DataTypeTok{byrow =} \OtherTok{TRUE}\NormalTok{, }\DataTypeTok{dimnames =} \KeywordTok{list}\NormalTok{(estados3b, estados3b));Mb}
\end{Highlighting}
\end{Shaded}

\begin{verbatim}
          0         1         2
0 0.3333333 0.3333333 0.3333333
1 0.3333333 0.3333333 0.3333333
2 0.3333333 0.3333333 0.3333333
\end{verbatim}

\begin{Shaded}
\begin{Highlighting}[]
\NormalTok{ProbT3b <-}\StringTok{ }\KeywordTok{new}\NormalTok{(}\StringTok{"markovchain"}\NormalTok{, }\DataTypeTok{states=}\NormalTok{estados3b, }\DataTypeTok{transitionMatrix=}\NormalTok{Mb)}
\KeywordTok{steadyStates}\NormalTok{(ProbT3b)}
\end{Highlighting}
\end{Shaded}

\begin{verbatim}
             0         1         2
[1,] 0.3333333 0.3333333 0.3333333
\end{verbatim}

Logo, verifica-se que para qualquer valor de y na distribuição
estacionária, seu valor dentro do vetor sempre será representado por
\(\pi(y) = \frac{1}{M+1}\), para todo \(y\epsilon S\). Ou seja, para 2
estados, \(M = 1\), e \(\pi(y)=\frac{1}{1+1}=0.5\). Para 3 estados,
\(M = 2\) e \(\pi(y) = 1/3\) e assim sucessivamente.

\pagebreak

\hypertarget{exercicio-4}{%
\subsubsection{Exercicio 4}\label{exercicio-4}}

Considere uma cadeia de Markov com espaço de estados \(S\) = \{0,1,2\} e
matriz de probabilidade de transição

\[\Gamma = 
\begin{matrix}
0\\ 
1\\ 
2
\end{matrix}
\begin{pmatrix}
0.4 & 0.4 & 0.2  \\
0.3 & 0.4 & 0.3  \\
0.2 & 0.4 & 0.4  
\end{pmatrix}\]

Mostre que esta cadeia tem uma única distribuição estacionária \(\pi\) e
encontre-a.

\textbf{R:}

\[\text{como: } \sum_{x}\pi(x)p_{\text{x, y}} = \pi{(y)}\ \quad y \in \mathcal{S} \]
\[\sum_{x}{\pi(x)} = 1 \] Portanto, o seguinte sistema de equações
lineares, caso tenha solução, encontre a distribuição estacionária:

\[ \begin{cases}0.4 \pi(0) + 0.3\pi(1) + 0.2\pi(2) = \pi(0), \quad \text{equação (1)} \\
  0.4 \pi(0) + 0.4\pi(1) + 0.4\pi(2) = \pi(1), \quad \text{equação (2)}  \\
  0.2 \pi(0) + 0.3\pi(1) + 0.4\pi(2) = \pi(2), \quad \text{equação (3)}  
  \end{cases}\]

Resolvendo o sistema: equação(1) - equação(3)

\[\begin{array}{rcl} 0.2*\pi(0) + 0*\pi(0) -0.2*\pi(2) &=& \pi(0) -\pi(2)\\
  0.2*\pi(2) &=& 0.8*\pi(0) \\
  \pi(2) &=& \pi(0)
  \end{array} \]

equação(1) - equação(2) e tendo \(\pi(2) = \pi(0)\)
\[ \begin{array}{rcl} 0*\pi(0) - 0.1*\pi(0) -0.2*\pi(2) &=& \pi(0) -\pi(1)\\
  -0.1*\pi(1) - 0.2 * \pi(2) &=& \pi(2) - \pi(1) \\
  0.9*\pi(1)  &=& 1.2 \pi(2) \\
    \pi(1) &=& \frac{4}{3} \pi(2)  \end{array} \]

Tendo \(\pi(0) = 0.75\), \(\pi(1) = \pi(2)\) e \(\sum_{x}\pi(x) = 1\). A
distribuição estacionária \[\pi = 
\begin{pmatrix}
\frac{3}{10}& \frac{2}{5} &  \frac{3}{10} 
\end{pmatrix}\]

\pagebreak

\hypertarget{exercicio-8}{%
\subsubsection{Exercicio 8}\label{exercicio-8}}

Considere uma cadeia de Markov com espaço de estados \(S\) = \{0,1,2\} e
matriz de probabilidade de transição

\[\Gamma =\begin{matrix}
0\\ 
1\\ 
2
\end{matrix}
\begin{pmatrix}
0 & 0 & 1  \\
1 & 0 & 0  \\
1/2 & 1/2 & 0  
\end{pmatrix}\]

\textbf{a)} Mostre que esta é uma cadeia irredutível.

\textbf{R:}

\begin{Shaded}
\begin{Highlighting}[]
\NormalTok{gamma8 <-}\StringTok{ }\KeywordTok{matrix}\NormalTok{(}\KeywordTok{c}\NormalTok{(}\DecValTok{0}\NormalTok{,}\DecValTok{0}\NormalTok{,}\DecValTok{1}\NormalTok{,}\DecValTok{1}\NormalTok{,}\DecValTok{0}\NormalTok{,}\DecValTok{0}\NormalTok{,}\FloatTok{0.5}\NormalTok{,}\FloatTok{0.5}\NormalTok{,}\DecValTok{0}\NormalTok{), }\DataTypeTok{byrow=}\NormalTok{T, }\DataTypeTok{ncol =} \DecValTok{3}\NormalTok{, }\DataTypeTok{dimnames=}\KeywordTok{list}\NormalTok{(}\DecValTok{0}\OperatorTok{:}\DecValTok{2}\NormalTok{))}
\NormalTok{ProbT8 =}\StringTok{ }\KeywordTok{new}\NormalTok{(}\StringTok{"markovchain"}\NormalTok{, }\DataTypeTok{states=}\KeywordTok{as.character}\NormalTok{(}\DecValTok{0}\OperatorTok{:}\DecValTok{2}\NormalTok{), }\DataTypeTok{transitionMatrix=}\NormalTok{gamma8, }\DataTypeTok{name=}\StringTok{"Γ"}\NormalTok{)}
\end{Highlighting}
\end{Shaded}

Abaixo apresento a saída do \emph{summary} da matriz de transição:

\begin{Shaded}
\begin{Highlighting}[]
\KeywordTok{summary}\NormalTok{(ProbT8)}
\end{Highlighting}
\end{Shaded}

\begin{verbatim}
G  Markov chain that is composed by: 
Closed classes: 
0 1 2 
Recurrent classes: 
{0,1,2}
Transient classes: 
NONE 
The Markov chain is irreducible 
The absorbing states are: NONE
\end{verbatim}

Aqui temos que todos os estados se comunicam, logo a matriz é
irredutível.

\textbf{b)} Encontre o período.

\textbf{R:}

Aplicando a Def. 34 temos:

\(\Gamma^1:\)

\begin{Shaded}
\begin{Highlighting}[]
\NormalTok{ProbT8}
\end{Highlighting}
\end{Shaded}

\begin{verbatim}
G 
 A  3 - dimensional discrete Markov Chain defined by the following states: 
 0, 1, 2 
 The transition matrix  (by rows)  is defined as follows: 
    0   1 2
0 0.0 0.0 1
1 1.0 0.0 0
2 0.5 0.5 0
\end{verbatim}

\(\Gamma^2:\)

\begin{Shaded}
\begin{Highlighting}[]
\NormalTok{ProbT8}\OperatorTok{^}\DecValTok{2}
\end{Highlighting}
\end{Shaded}

\begin{verbatim}
G^2 
 A  3 - dimensional discrete Markov Chain defined by the following states: 
 0, 1, 2 
 The transition matrix  (by rows)  is defined as follows: 
    0   1   2
0 0.5 0.5 0.0
1 0.0 0.0 1.0
2 0.5 0.0 0.5
\end{verbatim}

\(\Gamma^3:\)

\begin{Shaded}
\begin{Highlighting}[]
\NormalTok{ProbT8}\OperatorTok{^}\DecValTok{3}
\end{Highlighting}
\end{Shaded}

\begin{verbatim}
G^3 
 A  3 - dimensional discrete Markov Chain defined by the following states: 
 0, 1, 2 
 The transition matrix  (by rows)  is defined as follows: 
     0    1   2
0 0.50 0.00 0.5
1 0.50 0.50 0.0
2 0.25 0.25 0.5
\end{verbatim}

\(\Gamma^4:\)

\begin{Shaded}
\begin{Highlighting}[]
\NormalTok{ProbT8}\OperatorTok{^}\DecValTok{4}
\end{Highlighting}
\end{Shaded}

\begin{verbatim}
G^4 
 A  3 - dimensional discrete Markov Chain defined by the following states: 
 0, 1, 2 
 The transition matrix  (by rows)  is defined as follows: 
     0    1    2
0 0.25 0.25 0.50
1 0.50 0.00 0.50
2 0.50 0.25 0.25
\end{verbatim}

\(\Gamma^5:\)

\begin{Shaded}
\begin{Highlighting}[]
\NormalTok{ProbT8}\OperatorTok{^}\DecValTok{5}
\end{Highlighting}
\end{Shaded}

\begin{verbatim}
G^5 
 A  3 - dimensional discrete Markov Chain defined by the following states: 
 0, 1, 2 
 The transition matrix  (by rows)  is defined as follows: 
      0     1    2
0 0.500 0.250 0.25
1 0.250 0.250 0.50
2 0.375 0.125 0.50
\end{verbatim}

\(\Gamma^6:\)

\begin{Shaded}
\begin{Highlighting}[]
\NormalTok{ProbT8}\OperatorTok{^}\DecValTok{6}
\end{Highlighting}
\end{Shaded}

\begin{verbatim}
G^6 
 A  3 - dimensional discrete Markov Chain defined by the following states: 
 0, 1, 2 
 The transition matrix  (by rows)  is defined as follows: 
      0     1     2
0 0.375 0.125 0.500
1 0.500 0.250 0.250
2 0.375 0.250 0.375
\end{verbatim}

\(d_0=m.d.c \left \{ 2,3,4, \cdots \right \}=1\\\)
\(d_1=m.d.c \left \{ 3,5,6, \cdots \right \}=1\\\)
\(d_2=m.d.c \left \{ 2,3,4, \cdots \right \}=1\\\)

Do teorema 44, observamos que
\(\lim_{n\rightarrow \infty }\Gamma^{(n)} =\) \(\Gamma^6:\)

\begin{Shaded}
\begin{Highlighting}[]
\NormalTok{ProbT8}\OperatorTok{^}\DecValTok{50000}
\end{Highlighting}
\end{Shaded}

\begin{verbatim}
G^50000 
 A  3 - dimensional discrete Markov Chain defined by the following states: 
 0, 1, 2 
 The transition matrix  (by rows)  is defined as follows: 
    0   1   2
0 0.4 0.2 0.4
1 0.4 0.2 0.4
2 0.4 0.2 0.4
\end{verbatim}

Então temos que: \(\Pi(0)=0.4,\ \Pi(1)=0.2\ e\ \Pi(2)=0.4\\\)

\(\lim_{n\rightarrow \infty }\gamma^{(n)}_{x,0} = \pi(0) = 0.4\\\)
\(\lim_{n\rightarrow \infty }\gamma^{(n)}_{x,1} = \pi(1) = 0.2\\\)
\(\lim_{n\rightarrow \infty }\gamma^{(n)}_{x,2} = \pi(2) = 0.4\\\)

O que leva a conclusão de que a cadeia é aperiódica (\(d = 1\))

\textbf{c)} Encontre a distribuição estacionária.

\textbf{R:}

A distribuição estacionária é dada por:

\begin{Shaded}
\begin{Highlighting}[]
\NormalTok{estados4 <-}\StringTok{ }\KeywordTok{c}\NormalTok{(}\StringTok{"0"}\NormalTok{, }\StringTok{"1"}\NormalTok{, }\StringTok{"2"}\NormalTok{)}
\NormalTok{estados8 <-}\StringTok{ }\NormalTok{estados4}
\NormalTok{y <-}\StringTok{ }\KeywordTok{matrix}\NormalTok{(}\DataTypeTok{data=}\KeywordTok{c}\NormalTok{(}\DecValTok{0}\NormalTok{, }\DecValTok{0}\NormalTok{, }\DecValTok{1}\NormalTok{, }\DecValTok{1}\NormalTok{, }\DecValTok{0}\NormalTok{, }\DecValTok{0}\NormalTok{, }\FloatTok{0.5}\NormalTok{, }\FloatTok{0.5}\NormalTok{, }\DecValTok{0}\NormalTok{), }\DataTypeTok{nrow =} \DecValTok{3}\NormalTok{, }\DataTypeTok{ncol =} \DecValTok{3}\NormalTok{, }\DataTypeTok{byrow =} \OtherTok{TRUE}\NormalTok{,}
\DataTypeTok{dimnames =} \KeywordTok{list}\NormalTok{(estados8, estados8))}
\NormalTok{ProbT8 <-}\StringTok{ }\KeywordTok{new}\NormalTok{(}\StringTok{"markovchain"}\NormalTok{, }\DataTypeTok{states=}\NormalTok{estados8, }\DataTypeTok{transitionMatrix=}\NormalTok{y)}
\KeywordTok{steadyStates}\NormalTok{(ProbT8)}
\end{Highlighting}
\end{Shaded}

\begin{verbatim}
       0   1   2
[1,] 0.4 0.2 0.4
\end{verbatim}

Logo, a distribuição estacionária da cadeia acima é dada por
\(\pi_{1} = (0.4, 0.2, 0.4)\)

\pagebreak

\hypertarget{exercicio-12}{%
\subsubsection{Exercicio 12}\label{exercicio-12}}

Sejam \(\pi_{0}\) e \(\pi_{1}\) duas distribuições estacionárias
distintas para uma Cadeia de Markov.

\textbf{a)} Prove que para \(0\leq\alpha\leq\), a função
\(\pi_{\alpha}\) definida como:

\(\pi_{\alpha}(x)=(1-\alpha)\pi_{0}(x)+\alpha \pi_{1}(x)\),
\(x\epsilon S\), é uma distribuição estacionária.

\textbf{R:}

Nós temos que \(\pi_{0}\) e \(\pi_{1}\) são distribuições estacionárias
e, portanto, satisfazem a definição 26, ou seja:

\[\sum_{n}\pi(x)\gamma_{x,y}=\pi(y),y\epsilon S\]

Logo, podemos escrever que:

\[\sum_{n}\pi(x)\gamma_{x,y}=\sum_{n}(\alpha \pi_{1}(x)+(1-\alpha)\pi_{0}(x))\gamma_{x,y}=\alpha \pi_{1}(y)+(1-\alpha)\pi_{0}(y)=\pi\alpha(y)\]
Essa solução vale para todo y 2 S. A condição de que
\(0 \leq \pi\alpha \leq 1, \forall \epsilon S\) também procede e,
portanto, \(\pi_{\alpha}\) é uma distribuição estacionária.

\textbf{b)} Mostre que distintos valores de \(\alpha\) implicam em
distribuições estacionárias \(\pi_{\alpha}\) distintas. Para demonstrar
isso sugere-se escolher \(x_{0} \epsilon S\) tal que
\(\pi_{0}(x_{0})\neq\pi_{1}(x_{0})\) e prove que
\(\pi_{\alpha}(x_{0})=\pi_{\beta}(x_{0})\) implica que o
\(\alpha = \beta\).

\textbf{R:}

Sabemos que \(\pi_{0}\) e \(\pi_{1}\) são distintos, e, portanto, há um
valor \(x_{0} \epsilon S\) tal que
\(\pi_{0}(x_{0})\neq \pi_{1}(x_{0})\). Tomamos um valor \(\alpha\) e
\(\beta\) arbitrários do intervalo {[}0, 1{]}, que satisfaçam
\(\pi_{\alpha}(x_{0})\neq \pi_{\beta}(x_{0})\) para todo
\(x \epsilon S\). Logo, o que devemos mostrar aqui é que
\(\alpha = \beta\). Como a condição também é válida para \(x_{0}\),
temos que:

\[\alpha pi_{1}(x_{0})+(1- \alpha)\pi_{0}(x_{0})=\beta \pi_{1}(x_{0})+(1-\beta)\pi_{0}x_{0}\]

\[(\alpha -\beta) \pi_{1}(x_{0})+(-\alpha +\beta)\pi_{0}x_{0}\]
\[(\alpha -\beta) (\pi_{1}(x_{0})-\pi_{0}(x_{0}))=0\]

Como sabemos que \(\pi_{0}(x_{0})\neq \pi_{1}(x_{0})\), logo a igualdade
\(\alpha=\beta\), o que prova que distintos valores de \(\alpha\)
implicam distribuição estacionárias \(\pi_{\alpha}\) distintas.

\end{document}
