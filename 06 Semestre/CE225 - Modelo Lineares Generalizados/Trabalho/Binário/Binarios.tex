\documentclass[a4paper]{article} %% Classe do documento
\usepackage[margin=2cm]{geometry} %% Dimens?es das margens
\usepackage[brazil]{babel} %% Idioma
\usepackage[utf8]{inputenc} %% Codifica??o de caracteres
\usepackage{amsmath} %% S?mbolos/express?es/ambientes matem?ticos
\usepackage{graphicx} %% Inserir figuras em qualquer extens?o
\usepackage[svgnames]{xcolor} %% Usar cores pelo nome
\usepackage{float}
\usepackage{bm}
\usepackage{aeguill}
\usepackage{multicol}
\usepackage{indentfirst}
\usepackage {documentclass}
\usepackage{titling}

\newcommand{\undertilde}[1]{\underset{\widetilde{}}{#1}}

\begin{document}
\SweaveOpts{concordance=TRUE}
\begin{titlepage}

  \center{\rule{15cm}{2pt}}
  \begin{center}{\bf Universidade Federal do Parana\\
      Departamento de Estat\’{\i}stica\\[7.5cm]

      {\large
        ANÁLISE DE SOVREVIVENTES EM ACIDENTES DE CARRO}\\[3cm]

      { CE225 - Modelos Lineares Generalizados}\\[2cm]

      { La\’{\i}s Hoffmam }
      
      { Simone Matsubara }
      
      { Willian Meira }
      
      { Yasmin Fernandes }

      % \end{minipage}
      \vfill
      Curitiba, 14 de novembro de 2018
      \center{\rule{15cm}{2pt}}}
  \end{center}
\end{titlepage}

\tableofcontents
\pagebreak

\section{Introducao}
Os dados foram retirados do pacote "DAAG", sendo dados dos EUA, entre 1997-2002, de acidentes de carro relatados pela polícia nos quais há um evento prejudicial (pessoas ou propriedade) e do qual pelo menos um veículo foi rebocado.
Os dados são restritos aos ocupantes do banco da frente, incluem apenas um subconjunto das variáveis registradas e são restritos de outras maneiras também.


A base original possui uma base de dados com 26.217 observações nas 15 variáveis a seguir.

\section{Material e Metodos}
\subsection{Material}
A análise foi realizada com amostra de 5000 das 26.217 observações da base real, dos acidentes de carro.  Inicialmente tivemos acesso a 15 variavéis possivelmente explicativas. São elas:\\


\noindent\textbf{Veloc}: Velocidades estimadas do impacto do acidente\\
\textbf{Pesos}: Pesos de observação\\
\textbf{Sovrev}: Classificação se sobreviveu ao acidente\\
\textbf{Airbag}: Se o carro possui airbag\\
\textbf{Cinto}: Uso do cinto de segurança\\
\textbf{Frontal}: Impacto do acidente\\
\textbf{Sexo}: Sexo dos ocupantes do veículo\\
\textbf{Idade}: Idade dos ocupantes do veículo\\
\textbf{AnoAci}: Ano do acidente (1997-2002)\\
\textbf{AnoVei}: Ano do veículo (1953-2003)\\
\textbf{AbCat}: Se Airbags foram acionados\\
\textbf{Ocupantes}: Posição do airbag acionado\\
\textbf{Abfunc}: Airbag acionados\\
\textbf{Grav}: Gravidade do acidente\\
\textbf{Numcaso}: Número do caso\\


<<eval=TRUE,echo=FALSE,warning=FALSE,message=FALSE>>=

require(lattice)
require(latticeExtra)
require(xtable)


@

\begin{table}[ht]
\centering
\caption{Primeiras 6 observações do conjunto de dados}
\begin{tabular}{rrrrrrrrrrr}
  \hline
 & X1 & X2 & X3 & X4 & X5 & X6 & X7 & X8 & X9 & X10 \\ 
  \hline
1 & 7.00 & 0.27 & 0.36 & 20.70 & 0.04 & 45.00 & 170.00 & 1.00 & 3.00 & 0.45 \\ 
  2 & 6.30 & 0.30 & 0.34 & 1.60 & 0.05 & 14.00 & 132.00 & 0.99 & 3.30 & 0.49 \\ 
  3 & 8.10 & 0.28 & 0.40 & 6.90 & 0.05 & 30.00 & 97.00 & 1.00 & 3.26 & 0.44 \\ 
  4 & 7.20 & 0.23 & 0.32 & 8.50 & 0.06 & 47.00 & 186.00 & 1.00 & 3.19 & 0.40 \\ 
  5 & 7.20 & 0.23 & 0.32 & 8.50 & 0.06 & 47.00 & 186.00 & 1.00 & 3.19 & 0.40 \\ 
  6 & 8.10 & 0.28 & 0.40 & 6.90 & 0.05 & 30.00 & 97.00 & 1.00 & 3.26 & 0.44 \\ 
   \hline
\end{tabular}
\end{table}

\pagebreak

\subsection{Métodos}

Para conhecermos melhor os dados, apresentares algumas estatísticas. Assim podemos observar algumas medidas.

<<echo=FALSE>>=



@

\pagebreak

Dada as medidas descritivas acima observadas e considerando um alpha = 0.05, temos que as variáveis X3 e X8 são não significativas.

\section{Seleção de variáveis}

Com isso, utilizaremos o método Forward, com escala zero, para escontrar as melhores variáveis a serem utilizadas com certa ordenação. Juntamente, a ideia deste método é adicionar uma variável de cada vez, sendo a primeira variável selecionada aquela com maior correlação com a resposta, ou seja, ajusta um modelo que se adequa bem aos dados utilizando o calculo do AIC. No nosso caso, o modelo que possui o menor AIC é o seguinte:
\\

y= $ \beta_{0} + \beta_{1} x_{Acidezfixa} + \beta_{2} x_{Acidezvolátil} + \beta_{3} x_{Cloretos} + \beta_{4} x_{Dioxidototal} + \beta_{5} x_{Dioxidolivre} + \beta_{6} x_{PH} + \beta_{7} x_{Sulfatos} $
\\

Em seguida, fizemos novamente a análise de variância, para testar se as todas as variáveis eram significativas.

<<eval=TRUE,echo=FALSE,warning=FALSE,message=FALSE,include=FALSE>>=



@

<<>>=

@

\pagebreak

\section{Análise de Diagnóstico}

\subsection{Pressupostos}

<<eval=TRUE,echo=FALSE,warning=FALSE>>=



@

Observamos que no gráfico de análise de resíduos, os pontos estão pouco dispersos e com a linha de tendência constante em torno de zero (Atenção para a escala). Pressupomos também que existe normalidade. E aplicando o teste de Shapiro, confirmamos a normalidade, dado que o p-valor está abaixo 0.05.
\pagebreak


\section{Conclusão}

Este estudo mostra que as variáveis X3: Ácido cítrico, X4: Açúcar residual e X8: Densidade, não influenciaram na qualidade do vinho. As únicas variáveis significativas foram as implementadas no modelo. No começo do estudo haviam muitas variáveis para se trabalhar, e se deixassemos um modelo com muitas haveria dificuldade em encontrar uma que realmente influenciasse, então com a seleção feita, utilizando o método AIC, utilizamos apenas variáveis que eram significativas na Qualidade final do vinho.


\end{document}