\documentclass[]{article}
\usepackage{lmodern}
\usepackage{amssymb,amsmath}
\usepackage{ifxetex,ifluatex}
\usepackage{fixltx2e} % provides \textsubscript
\ifnum 0\ifxetex 1\fi\ifluatex 1\fi=0 % if pdftex
  \usepackage[T1]{fontenc}
  \usepackage[utf8]{inputenc}
\else % if luatex or xelatex
  \ifxetex
    \usepackage{mathspec}
  \else
    \usepackage{fontspec}
  \fi
  \defaultfontfeatures{Ligatures=TeX,Scale=MatchLowercase}
\fi
% use upquote if available, for straight quotes in verbatim environments
\IfFileExists{upquote.sty}{\usepackage{upquote}}{}
% use microtype if available
\IfFileExists{microtype.sty}{%
\usepackage{microtype}
\UseMicrotypeSet[protrusion]{basicmath} % disable protrusion for tt fonts
}{}
\usepackage[margin=1in]{geometry}
\usepackage{hyperref}
\hypersetup{unicode=true,
            pdftitle={Trabalho de dados Binários},
            pdfauthor={Laís Hoffmann, Simone Matsubara, Yasmin Fernandes, Willian Meira},
            pdfborder={0 0 0},
            breaklinks=true}
\urlstyle{same}  % don't use monospace font for urls
\usepackage{graphicx,grffile}
\makeatletter
\def\maxwidth{\ifdim\Gin@nat@width>\linewidth\linewidth\else\Gin@nat@width\fi}
\def\maxheight{\ifdim\Gin@nat@height>\textheight\textheight\else\Gin@nat@height\fi}
\makeatother
% Scale images if necessary, so that they will not overflow the page
% margins by default, and it is still possible to overwrite the defaults
% using explicit options in \includegraphics[width, height, ...]{}
\setkeys{Gin}{width=\maxwidth,height=\maxheight,keepaspectratio}
\IfFileExists{parskip.sty}{%
\usepackage{parskip}
}{% else
\setlength{\parindent}{0pt}
\setlength{\parskip}{6pt plus 2pt minus 1pt}
}
\setlength{\emergencystretch}{3em}  % prevent overfull lines
\providecommand{\tightlist}{%
  \setlength{\itemsep}{0pt}\setlength{\parskip}{0pt}}
\setcounter{secnumdepth}{0}
% Redefines (sub)paragraphs to behave more like sections
\ifx\paragraph\undefined\else
\let\oldparagraph\paragraph
\renewcommand{\paragraph}[1]{\oldparagraph{#1}\mbox{}}
\fi
\ifx\subparagraph\undefined\else
\let\oldsubparagraph\subparagraph
\renewcommand{\subparagraph}[1]{\oldsubparagraph{#1}\mbox{}}
\fi

%%% Use protect on footnotes to avoid problems with footnotes in titles
\let\rmarkdownfootnote\footnote%
\def\footnote{\protect\rmarkdownfootnote}

%%% Change title format to be more compact
\usepackage{titling}

% Create subtitle command for use in maketitle
\newcommand{\subtitle}[1]{
  \posttitle{
    \begin{center}\large#1\end{center}
    }
}

\setlength{\droptitle}{-2em}

  \title{Trabalho de dados Binários}
    \pretitle{\vspace{\droptitle}\centering\huge}
  \posttitle{\par}
  \subtitle{Acidentes de carro}
  \author{Laís Hoffmann, Simone Matsubara, Yasmin Fernandes, Willian Meira}
    \preauthor{\centering\large\emph}
  \postauthor{\par}
      \predate{\centering\large\emph}
  \postdate{\par}
    \date{2018-11-07}

\usepackage[brazil]{babel} \usepackage{amsmath} \usepackage{float}
\usepackage{bm}

\begin{document}
\maketitle

\section{1. Base de Dados}\label{base-de-dados}

\section{2 Análise Descritiva}\label{analise-descritiva}

\textbf{2.1 Medidas de Resumo} \newline **2.2 Boxplots** \newline **2.3
Histogramas** \newline **2.4 Distribuição** \newline **2.5 Análise de
correlações entre covariáveis** \newline **2.6 Gráficos de Disperção**

\section{3. AJUSTE DO MODELO DE
REGRESSÃO}\label{ajuste-do-modelo-de-regressao}

\textbf{3.1 Ligação Logito} \newline **3.2 Ligação Probito**
\newline **3.3 Ligação Complemento log-log** \newline **3.4 Ligação
Cauchy**

\section{4. ESCOLHA DO MODELO}\label{escolha-do-modelo}

\section{5. ANÁLISE DO MODELO AJUSTADO
SELECIONADO}\label{analise-do-modelo-ajustado-selecionado}

\textbf{5.1 Resumo do Modelo} \newline **5.2 Reajuste do Modelo**
\newline **5.3 Análise de Resíduos** \newline **5.4 Medidas de
Influencia** \newline **5.5 Resíduos Quantílicos Aleatoriazados**
\newline **5.6 Gráfico Normal de Probabilidade com Envelope Simulado**
\newline **5.7 Gráficos de Efeitos**

\section{6. PREDIÇÃO}\label{predicao}

\section{7. AVALIAÇÃO DO PODER PREDITIVO DO
MODELO}\label{avaliacao-do-poder-preditivo-do-modelo}

\textbf{7.1 Divisão da Base de dados} \newline **7.2 Ponto de Corte**
\newline **7.3 Sensibilidade e Especificidade** \newline **7.4 Curva
ROC** \newline **7.5 Outra Alternativa de validação**

\section{8. REFERÊNCIAS}\label{referencias}

\section{}\label{section}


\end{document}
