% Options for packages loaded elsewhere
\PassOptionsToPackage{unicode}{hyperref}
\PassOptionsToPackage{hyphens}{url}
%
\documentclass[
]{article}
\usepackage{lmodern}
\usepackage{amssymb,amsmath}
\usepackage{ifxetex,ifluatex}
\ifnum 0\ifxetex 1\fi\ifluatex 1\fi=0 % if pdftex
  \usepackage[T1]{fontenc}
  \usepackage[utf8]{inputenc}
  \usepackage{textcomp} % provide euro and other symbols
\else % if luatex or xetex
  \usepackage{unicode-math}
  \defaultfontfeatures{Scale=MatchLowercase}
  \defaultfontfeatures[\rmfamily]{Ligatures=TeX,Scale=1}
\fi
% Use upquote if available, for straight quotes in verbatim environments
\IfFileExists{upquote.sty}{\usepackage{upquote}}{}
\IfFileExists{microtype.sty}{% use microtype if available
  \usepackage[]{microtype}
  \UseMicrotypeSet[protrusion]{basicmath} % disable protrusion for tt fonts
}{}
\makeatletter
\@ifundefined{KOMAClassName}{% if non-KOMA class
  \IfFileExists{parskip.sty}{%
    \usepackage{parskip}
  }{% else
    \setlength{\parindent}{0pt}
    \setlength{\parskip}{6pt plus 2pt minus 1pt}}
}{% if KOMA class
  \KOMAoptions{parskip=half}}
\makeatother
\usepackage{xcolor}
\IfFileExists{xurl.sty}{\usepackage{xurl}}{} % add URL line breaks if available
\IfFileExists{bookmark.sty}{\usepackage{bookmark}}{\usepackage{hyperref}}
\hypersetup{
  pdftitle={Trabalho de dados Binários},
  pdfauthor={Laís Hoffmam, Simone Matsubara, Yasmin Fernandes, Willian Meira},
  hidelinks,
  pdfcreator={LaTeX via pandoc}}
\urlstyle{same} % disable monospaced font for URLs
\usepackage[margin=1in]{geometry}
\usepackage{graphicx,grffile}
\makeatletter
\def\maxwidth{\ifdim\Gin@nat@width>\linewidth\linewidth\else\Gin@nat@width\fi}
\def\maxheight{\ifdim\Gin@nat@height>\textheight\textheight\else\Gin@nat@height\fi}
\makeatother
% Scale images if necessary, so that they will not overflow the page
% margins by default, and it is still possible to overwrite the defaults
% using explicit options in \includegraphics[width, height, ...]{}
\setkeys{Gin}{width=\maxwidth,height=\maxheight,keepaspectratio}
% Set default figure placement to htbp
\makeatletter
\def\fps@figure{htbp}
\makeatother
\setlength{\emergencystretch}{3em} % prevent overfull lines
\providecommand{\tightlist}{%
  \setlength{\itemsep}{0pt}\setlength{\parskip}{0pt}}
\setcounter{secnumdepth}{-\maxdimen} % remove section numbering
\usepackage[brazil]{babel} \usepackage{amsmath} \usepackage{float} \usepackage{bm}

\title{Trabalho de dados Binários}
\usepackage{etoolbox}
\makeatletter
\providecommand{\subtitle}[1]{% add subtitle to \maketitle
  \apptocmd{\@title}{\par {\large #1 \par}}{}{}
}
\makeatother
\subtitle{Acidentes de carro}
\author{Laís Hoffmam, Simone Matsubara, Yasmin Fernandes, Willian Meira}
\date{2020-09-14}

\begin{document}
\maketitle

´´´\{r, include=FALSE\} library(lattice) library(readxl) library(readr)
library(tidyverse) library(MASS) library(DAAG) library(gridExtra)
library(corrplot) library(carData) library(car) library(statmod)
library(effects) library(ROCR) library(hnp) library(faraway)
library(ggplot2) ´´´

\hypertarget{base-de-dados}{%
\section{1. Base de Dados}\label{base-de-dados}}

\emph{1.1 Descrição dos dados}

Os dados foram retirados do pacote ``DAAG'', sendo dados dos EUA, entre
1997-2002, de acidentes de carro relatados pela polícia nos quais há um
evento prejudicial (pessoas ou propriedade) e do qual pelo menos um
veículo foi rebocado. Os dados são restritos aos ocupantes do banco da
frente, incluem apenas um subconjunto das variáveis registradas e são
restritos de outras maneiras também.

A base original possui uma base de dados com 26.217 observações nas 15
variáveis a seguir.

1 - \emph{veloc}: velocidades estimadas do impacto do acidente: 1-9km/h,
10-24, 25-39, 40-54, 55+ \newline 2 - \emph{pesos}: Pesos de observação
\newline 3 - \emph{sobrev}: Classificação se sobreviveu ao acidente: 1 =
sobreviveu ou 0 = morreu \newline 4 - \emph{airbag}: Se o carro possui
airbag: com ou sem airbag \newline 5 - \emph{cinto}: uso do cinto de
segurança: com ou sem cinto \newline 6 - \emph{frontal}: impacto do
acidente: 0 = não frontal, 1 = impacto frontal \newline 7 - \emph{sexo}:
Sexo: 0 = Feminino ou 1 = Masculino \newline 8 - \emph{idade}: Idade dos
ocupantes do veículo \newline 9 - \emph{anoaci}: Ano do acidente
(1997-2002) \newline 10 - \emph{anovei}: Ano do veículo (1953-2003)
\newline 11 - \emph{airbagcat}: Se Airbags foram acionados: deploy,
nodeploy, unavail \newline 12 - \emph{ocupantes}: Posição do airbag
acionado: driver, pass \newline 13 - \emph{abfunc}: Airbag acionados: 0:
Se não possuia airbag ou não foi acionado, 1: Um ou mais airbags foram
acionados \newline 14 - \emph{grav}: Gravidade do acidente: 0:none, 1 =
Possível Lesão, 2:no incapacity, 3:incapacity, 4:killed; 5:unknown,
6:prior death \newline 15 - \emph{numcaso}: Número do caso.

O escopo da análise tem como variável respota a sobrevivencia após o
acidente e ás demais variáveis serão as covariáveis explicativas.

Antes da analise começar foi verificado que algumas das variáveis
presentes na base são irrelevantes para o modelo.

As variáveis são: anoaci: ano do acidente anovei: ano do veiculo peso
das observações: sem descrição airbagcat e abfunc: pois ja temos na base
numcaso: id grav

\{r, include=FALSE\} \#\# Carregando e ajustando a base de dados
\#********************************* dados \textless-
read.csv(``C:\textbackslash Users\textbackslash Ketlin\textbackslash Desktop\textbackslash basevivomorto.csv'',
header = T, sep = `,') dados\textless-dados{[},c(-1,-10){]}
dados\(dvcat <- ifelse(dados\)dvcat == `1-9km/h',1,
ifelse(dados\(dvcat == '10-24',2,  ifelse(dados\)dvcat == `25-39',3,
ifelse(dados\(dvcat == '40-54',4,5)))) dados\)dvcat \textless-
as.factor(dados\$dvcat)

\hypertarget{anuxe1lise-descritiva}{%
\section{2 Análise Descritiva}\label{anuxe1lise-descritiva}}

\emph{2.1 Medidas de Resumo}

\{r\}
names(dados)\textless-c(`veloc',`sobrev',`cinto',`frontal',`sexo',`idade',`ocupantes',`airbag')
summary(dados)

Nota-se na varável velocidade uma frequência maior de acidentes na faixa
de 25-39 milhas. A maioria estava com cinto de segurança e os acidentes
foram a maioria frontais.

\emph{2.3 Histogramas} \{r\} x11() par(mfrow = c(1,3))
barplot(table(dados\(sobrev,dados\)cinto), beside=T, las = 1, xlab =
`Cinto', ylab = `Frequência', legend = c(`Não',`Sim'))
barplot(table(dados\(sobrev,dados\)frontal), beside=T, las = 1, xlab =
`Frontal', ylab = `Frequência', legend = c(`Não',`Sim'))
barplot(table(dados\(sobrev,dados\)sexo), beside=T, las = 1, xlab =
`Sexo', ylab = `Frequência', legend = c(`Não',`Sim'))

\hypertarget{ajuste-do-modelo-de-regressuxe3o}{%
\section{3. Ajuste do Modelo de
Regressão}\label{ajuste-do-modelo-de-regressuxe3o}}

\#\#\emph{3.1 Ligação Logito}

Vamos ajustar um Modelo Linear Generalizado Binomial com função de
ligação Logito. A expressão do modelo é dada por:

\(ln (\frac{\pi_i}{1-\pi_i}) = \beta_0 + \beta_1 Veloc_i + \beta_2 Sobrev_i+ \beta_4 Cinto_i + \beta_5 Frontal_i + \beta_6 Sexo_i + \beta_7 Idade_i + \beta_8 Ocupantes_i + \beta_9 airbag_i\)

No R, o modelo é declarado da seguinte forma:

\{r\} ajuste1 \textless- glm(sobrev \textasciitilde{}
.,family=binomial(link=`logit'),data = dados)

\#\#\emph{3.2 Ligação Probito}

Vamos ajustar um Modelo Linear Generalizado Binomial com função de
ligação Probito. A expressão do modelo é dada por:

\(\phi^{-1} (\pi_i) = \beta_0 + \beta_1 Veloc_i + \beta_4 Cinto_i + \beta_5 Frontal_i + \beta_6 Sexo_i + \beta_7 Idade_i + \beta_8 Ocupantes_i + \beta_9 airbag_i\)

No R, o modelo é declarado da seguinte forma:

\{r\} ajuste2 \textless- glm(sobrev \textasciitilde{}
.,family=binomial(link = `probit'),data = dados)

\#\#\emph{3.3 Ligação Complemento log-log}

Vamos ajustar um Modelo Linear Generalizado Binomial com função de
ligação Complemento Log Log. A expressão do modelo é dada por:

\(ln[-ln(1-\pi_i)] = \beta_0 + \beta_1 Veloc_i + \beta_4 Cinto_i + \beta_5 Frontal_i + \beta_6 Sexo_i + \beta_7 Idade_i + \beta_8 Ocupantes_i + \beta_9 airbag_i\)

No R, o modelo é declarado da seguinte forma:

\{r\} ajuste3 \textless- glm(sobrev \textasciitilde{}
.,family=binomial(link=`cloglog'),data = dados)

\#\#\emph{3.4 Ligação Cauchy}

Vamos ajustar um Modelo Linear Generalizado Binomial com função de
ligação Cauchy. A expressão do modelo é dada por:

\(tan[\pi_i(\mu_i- 0,5)] = \beta_0 + \beta_1 Veloc_i + \beta_4 Cinto_i + \beta_5 Frontal_i + \beta_6 Sexo_i + \beta_7 Idade_i + \beta_8 Ocupantes_i + \beta_9 airbag_i\)

No R, o modelo é declarado da seguinte forma:

\{r\} ajuste4 \textless- glm(sobrev \textasciitilde{}
.,family=binomial(link=`cauchit'),data = dados)

\hypertarget{escolha-do-modelo}{%
\section{4. Escolha do Modelo}\label{escolha-do-modelo}}

O critério de informação AIC pode também ser utilizado, porém o AIC
penaliza o número de parâmetros do modelo. Como os modelos tem o mesmo
número de parâmetros, o critério aponta para a mesma direção da
verossimilhança pois todos são penalizados da mesma forma.

\{r, echo=FALSE, eval=TRUE, results=``hide''\} selec \textless-
data.frame(ajuste=c(`logito', `probito', `cloglog', `cauchy'),
aic=c(AIC(ajuste1), AIC(ajuste2), AIC(ajuste3), AIC(ajuste4)),
logLik=c(logLik(ajuste1),logLik(ajuste2),logLik(ajuste3),logLik(ajuste4)))
selec

O modelo que apresentou menor AIC e maior verossimilhança foi o modelo
Binomial com função de ligação Cauchy.

\hypertarget{anuxe1lise-do-modelo-ajustado-selecionado}{%
\section{5. Análise do Modelo Ajustado
Selecionado}\label{anuxe1lise-do-modelo-ajustado-selecionado}}

\#\#\emph{5.1 Resumo do Modelo}

\{r, echo=FALSE, eval=TRUE, results=``hide''\} summary(ajuste4)

\#\#\emph{5.2 Reajuste do Modelo}

Como próximo passo será usado o algoritmo stepwise para seleção de
variáveis do modelo.

O novo modelo fica da seguinte forma:

\{r, echo=TRUE, eval=TRUE, results=``hide''\} ajuste4.1 \textless-
step(ajuste4, direction = ``both'')

\{r, echo=FALSE, eval=TRUE, results=``hide''\}

summary(ajuste4)

\{r, echo=FALSE, eval=TRUE, results=``hide''\}

summary(ajuste4.1)

Vamos testar a seguinte hipotese: \(H_0\): modelos não diferem \(H_1\):
modelos diferem

\{r\} anova(ajuste4, ajuste4.1, test = `Chisq') summary(ajuste4.1)

P-valor é de 0.26 os modelos não diferem ou seja airbag pode ser
retirado do modelo

O modelo final ficou da seguinte forma :

\$tan{[}\pi\_i(\mu\_i- 0,5){]} = \beta\_0 + \beta\_1 Veloc\_i + \beta\_4
Cinto\_i + \beta\_5 Frontal\_i + \beta\_6 Sexo\_i + \beta\_7 Idade\_i +
\beta\_8 Ocupantes\_i \$ \#\#\emph{5.3 Análise de Resíduos}

\{r, echo=FALSE, eval=TRUE, results=``hide''\}

par(mfrow=c(2,2)) plot(ajuste4.1, 1:4) sq par(mfrow=c(2,2))
plot(ajuste4, 1:4)

\emph{5.4 Medidas de Influencia}

\{r\} influenceIndexPlot(ajuste4.1, vars=c(``Cook''), main=``Distância
de Cook'')

\{r\} influenceIndexPlot(ajuste4.1, vars=c(``Studentized''),
main=``Resíduos Padronizados'')

\emph{5.5 Resíduos Quantílicos Aleatoriazados}

\#\#\emph{5.6 Gráfico Normal de Probabilidades com Envelope Simulado}

Lineu O gráfico de resíduos simulados permite verificar a adequação do
modelo ajustado mesmo que os resíduos não tenham uma aproximação
adequada com a distribuição Normal. Neste tipo de gráfico espera-se,
para um modelo bem ajustado, os pontos (resíduos) dispersos
aleatoriamente entre os limites do envelope.

Deve-se ficar atento à presença de pontos fora dos limites do envelope
ou ainda a pontos dentro dos limites porém apresentando padrões
sistemáticos.

Vamos utilizar a função envelope implementada pelo professor Cesar
Augusto Taconeli :

\{r, echo=FALSE, eval=TRUE, results=``hide''\} envSim \textless-
function(model, data, nsim = 100)\{ dados \textless- na.omit(data) n
\textless- .subset2(model, ``df.null'') + 1 resM \textless- matrix(0,
nrow = n, ncol = nsim) sim \textless- simulate(model, nsim) for (i in
1:nsim)\{ dados\$y \textless- .subset2(sim, i) mSim \textless-
update(model, y \textasciitilde{} ., data = dados) resM{[},i{]}
\textless- sort.default(rstandard(mSim, type = `deviance'), na.last =
TRUE) \} qS \textless- apply(resM, 1 , quantile, c(0.025, 0.5, 0.975),
na.rm = TRUE) qN \textless- qnorm((1:n-0.5)/n) plot(rep(qN, 2),
c(qS{[}1,{]}, qS{[}3,{]}), type = `n', xlab = `Percentil da N(0,1)',
ylab = `Resíduos Padronizados', main = `Gráfico Normal de
Probabilidades') lines(qN, qS{[}1,{]}, type = `l') lines(qN, qS{[}2,{]},
type = `l', lty = 2, col = 4) lines(qN, qS{[}3,{]}, type = `l')
points(qN, sort.default(rstandard(model, type = `deviance'), na.last =
TRUE), pch = 16, cex = 0.75) \}

envSim(model = ajuste4.1, data = ajuste4.1\$data)

\emph{5.7 Gráficos de Efeitos}

\{r, echo=FALSE, eval=TRUE, results=``hide''\}
plot(allEffects(ajuste4.1), type = `response', main = '')

\hypertarget{prediuxe7uxe3o}{%
\section{6. PREDIÇÃO}\label{prediuxe7uxe3o}}

\hypertarget{avaliauxe7uxe3o-do-poder-preditivo-do-modelo}{%
\section{7. AVALIAÇÃO DO PODER PREDITIVO DO
MODELO}\label{avaliauxe7uxe3o-do-poder-preditivo-do-modelo}}

Como temos uma base de tamanho razoável para fins preditivos, uma
alternativa é separar a base em duas: uma para o ajuste do modelo, com
70\% dos dados (com 477 observações) e outra para validação, com 30\%
(com 203 observações).

\hypertarget{divisuxe3o-da-base-de-dados}{%
\subsection{\texorpdfstring{\emph{7.1 Divisão da Base de
dados}}{7.1 Divisão da Base de dados}}\label{divisuxe3o-da-base-de-dados}}

\{r\} set.seed(1909) indices \textless- sample(1:680, size = 477)
dadosajuste \textless- dados{[}indices,{]} dadosvalid \textless-
dados{[}-indices,{]}

\hypertarget{ponto-de-corte}{%
\subsection{\texorpdfstring{\emph{7.2 Ponto de
Corte}}{7.2 Ponto de Corte}}\label{ponto-de-corte}}

Como estamos modelando a probabilidade de tumor maligno, vamos
estabelecer o ponto de corte 0.5, isso é, se a probabilidade estimada
for maior que este valor o tumor será classificado como maligno. Vamos
armazenar os valores preditos do modelo para os dados de validação:

\{r\} pred \textless- predict(ajuste4.1, newdata = dadosvalid, type =
`response') corte \textless- ifelse(pred \textgreater{} 0.5, `maligno',
`benigno')

\emph{7.3 Sensibilidade e Especificidade}

Para fazer uso dos dados de validação, dois conceitos são necessários:
sensibilidade e especificidade.

Define-se por sensibilidade a capacidade do modelo de detectar tumores
malignos, ou seja, de classificar como malignos os tumores que de fato o
são .

Já a especificidade é a capacidade do modelo de detectar classificar
como benignos tumores verdadeiramente benignos.

A sensibilidade é dada por

\{r\} sens \textless- dados{[}2,2{]}/sum(dados{[},2{]}) sens

\emph{7.4 Curva ROC}

\emph{7.5 Outra Alternativa de validação}

\hypertarget{referuxeancias}{%
\section{8. REFERÊNCIAS}\label{referuxeancias}}

\end{document}
